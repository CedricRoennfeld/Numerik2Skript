% !TeX root = ../script.tex

\section{Minimierung von Funktionen}
Wir erinnern uns an das Gradientenabstiegsverfahren, in welchem wir die Minimierung des Funktionals 
%
\begin{align*}
  \phi(x) = \dfrac{1}{2}x^HAx - x^Hb 
\end{align*}
%
verwendet haben um die Lösung des linearen Gleichungssystems $Ax=b$ zu finden. 

Analog dazu können wir diesen Ansatz auch für beliebige Nullstellenprobleme verwenden. 

Für eine hinreichend glatte Funktion $f:D(f)\subset \R^n\to\R^m$ mit $m\geq n$, 
wandeln wir unser Nullstellenproblem $f(x)=0$ in ein nichtlineares Ausgleichsproblem um und minimieren 
%
\begin{align*}
  \phi(x) = \dfrac{1}{2}\vertn{2}{f(x)}_2^2
\end{align*}

Für eine Nullstelle $\hat{x}$ von $f$ gelten die notwendigen und hinreichenden Bedingungen eines lokalen Minimums:
%
\begin{align*}
  \nabla \phi(\hat{x}) 
  &= J_f(\hat{x})^T \cdot f(\hat{x}) = 0\\
  \nabla^2 \phi(\hat{x}) 
  &= J_f(\hat{x})^TJ_f(\hat{x}) + \sum_{i=1}^{m} f_i(\hat{x})\cdot \nabla^2 f_i(\hat{x}) 
  = J_f(\hat{x})^TJ_f(\hat{x}) \succeq 0
\end{align*}
%

Um eine potentielle Nullstelle von $f$ zu finden können wir also auch ein Minimum von $\phi$ bestimmen. Das 
bekannte Gradientenabstiegsverfahren wäre eine Möglichkeit hierfür, konvergiert jedoch nicht besonders schnell.

Eine Alternative hierzu ist die Newton-Methode oder Newton-artige Verfahren, bei welchen wir eine in jedem 
Iterationsschritt Linearisierung durchführen.

Als kurze Wiederholung betrachten wir hierfür noch einmal die Idee hinter der Newton Methode. Gegeben sei erneut 
eine hinreichend glatte Funktion $f:\R^n\to\R^n$. Für das Ziel eine Nullstelle von $f$ zu finden, betrachten wir 
die Linearisierung (Taylor-Formel 1. Ordnung)
%
\begin{align*}
  f_L(x) = f(x_0) + \dfrac{\partial }{\partial x} f(x) \Big|_{x=x_0}\cdot (x-x_0)
\end{align*}
%
Da $f(x) \approx f_L(x)$ bestimmen wir nun die Nullstelle von $f_L(x)$. Als lineares Gleichungssystem ist dies leicht 
zu lösen uns es ergibt sich 
%
\begin{align*}
  x = x_0 - \left(\dfrac{\partial }{\partial x} f(x) \Big|_{x=x_0}\right)^{-1}\cdot F(x_0)
\end{align*}
%
Aus dieser Nullstelle bilden wir eine rekursive Vorschrift und erhalten die mehrdimensionale Newton-Methode:
%
\begin{align*}
  x_{k+1} = x_k - \left(\dfrac{\partial }{\partial x} f(x) \Big|_{x=x_k}\right)^{-1}\cdot F(x_k),
  \qquad x_0\in\R^n
\end{align*}
%
\textcolor{red}{Probleme von Newton + Konvergenzgeschwindigkeit}