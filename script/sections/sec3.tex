% !TeX root = ../script.tex
\section{Eigenwertprobleme}

\subsection{Einleitung}
Aus der linearen Algebra ist bereits das klassische Eigenwertproblem bekannt:

Gegeben sei eine Matrix $A\in\K^{n\times n}$ und gesucht sind 
$\lambda\in\K$ und $v\in\K^n$, $v\neq 0$ 
sodass 
%
\begin{align*}
  Av=\lambda v
\end{align*}

Das Umstellen des Eigenwertproblems ergibt das System 
%
\begin{align*}
  (A-\lambda I)v=0
  \tag{*}\label{eq:EWPlgs}
\end{align*}
Hierbei muss $A-\lambda I$ singulär sein, sonst ist die eindeutige Lösung des Systems gegeben durch $v=0$.

Per Hand würden wir hier nun das charakteristische Polynom $\chi_A(\lambda)=\det(A-\lambda I)$ aufstellen und 
dessen Nullstellen bestimmen, da dies genau die Werte für $\lambda$ sind, für welche das System (\ref{eq:EWPlgs}) 
nicht-triviale Lösungen hat. 

Für die numerische Berechnung der Eigenwerte ist dies nicht ratsam, da Nullstellenbestimmung bei Polynomen 
hochgradig schlecht konditioniert ist. 

Wir stellen folgende Zusammenhänge der Berechnung von Eigenwerten und Eigenvektoren fest:

\begin{enumerate}
  \item[a)] 
    Eigenwert bekannt $\implies$ Eigenvektor als Lösung eines LGS (*).
  \item[b)] 
    Eigenvektor bekannt $\implies$ Eigenwert über Rayleigh-Quotient 
    $\lambda=\dfrac{\langle Av,v\rangle}{\vertn{2}{v}_2^2}$ 
\end{enumerate}

\subsection{Einschließungssätze und Stabilität}

\begin{colbox}{Hilfssatz}\label{hsatz:normIneq}
  Seien $A,B\in\K^{n\times n}$ beliebige Matrizen und $\vertn{2}{\cdot}_{\op}$ eine durch $\vertn{2}{\cdot}$
  induzierte Operatornorm. 
  Dann gilt für jeden Eigenwert $\lambda$ von $A$, welcher nicht zugleich auch Eigenwert von $B$ ist, die Beziehung
  %
  \begin{align*}
    \vertn{2}{(\lambda I-B)^{-1}(A-B)}_{\op}\geq 1
  \end{align*}
  %
\end{colbox}

\textit{Beweis.} \\
Ist $w$ ein Eigenvektor vom Eigenwert $\lambda$ von $A$, so folgt aus der Identität 
$(A-B)w = (I\lambda - B)w$, wenn $\lambda$ kein Eigenwert von $B$ ist, d.h. $\lambda I-B$ invertierbar, dass:
% 
\begin{align*}
  (\lambda I-B)^{-1}(A-B)w = w
\end{align*}
%
Demnach ist also 
%
\begin{align*}
  1\leq \sup_{x\in\K^n\backslash\{0\}} \dfrac{\vertn{2}{(\lambda I-B)^{-1}(A-B)x}}{\vertn{2}{x}}
  =\vertn{2}{(\lambda I-B)^{-1}(A-B)}_{\op}
\end{align*}
%
\subsubsection{Gerschgorin-Kreise}
%
\begin{colbox}{Satz}[Satz von Gerschgorin]\label{satz:gerschgorin}
  Alle Eigenwerte eine Matrix $A\in\K^{n\times n}$ liegen in der Vereinigung der sogenannten 
  Gerschgorin-Kreise
  %
  \begin{align*}
    K_j := \left\{z\in\C : |z-a_{jj}|\leq \sum_{k\neq j} |a_{jk}|\right\},
    \qquad \text{für } j=1,\dots,n
  \end{align*}
  %
  Für eine Teilmenge $I\subset\{1,\dots,n\}$ gilt, sind die Mengen $U=\displaystyle \bigcup_{j\in I} K_j$ 
  und $V=\displaystyle \bigcup_{j\notin I} K_j$ disjunkt, so liegen in $U$ genau $m:=|I|$ und in $V$ genau $n-m$ 
  Eigenwerte von $A$ (mehrfache Eigenwerte werden entsprechend ihrer algebraischen Vielfachheit gezählt).
\end{colbox}

\textit{Beweis.} \\
Zur ersten Behauptung: \\
Wir setzen $B=D:=\diag(a_{jj})$ in dem Hilfssatz \ref{hsatz:normIneq} und nehmen 
$\vertn{2}{\cdot}_\infty$ als natürliche Matrixnorm. 

Für $\lambda\neq a_{jj}$ folgt dann 
%
\begin{align*}
  \vertn{2}{(\lambda I -D)^{-1}(A-D)}_\infty 
  = \max_{j=1,\dots,n}\dfrac{1}{|\lambda-a_{jj}|}\sum_{k\neq j} |a_{jk}|
  \geq 1,
\end{align*}
%
d.h. $\lambda$ liegt in einem der Gerschgorin-Kreise. 

Für den zweiten Teil setzen wir $A_t = D + t(A-D)$, 
dann hat $A_0 = D$ als Eigenwerte $a_{11},\dots,a_{nn}$ die Mittelpunkte von $K_1,\dots,K_n$ 
und damit liegen genau $m$ dieser Eigenwerte in $U$ und die restlichen $n-m$ in $V$. 

Die Abbildung $t\mapsto D + t(A-D)$ 
ist stetig in $t$ und da Eigenwerte einer Matrix selbst stetig sind(d.h. ändert sich die Matrix leicht ab, dann ändern 
sich auch die Eigenwerte nur leicht ab), folgt, dass beim Wandern von $t=0$ zu $t=1$ und damit von $A_0=D$ zu $A_1=A$ sich 
die Eigenwerte stetig durch den Raum $\K$ bewegen. Da die Mengen $U$ und $V$ abgeschlossen und disjunkt sind haben sie einen 
positiven Abstand, welcher durch die stetige Bewegung der Eigenwerte nicht überschritten werden kann.
\qed 

Ein Alternativer Beweis zur ersten Behauptung liefert eine Betrachtung des 
Eigenwertproblems $Ax=\lambda x$ mit Eigenvektor $x\neq 0$. 
Offensichtlich existiert ein $x_i$ mit $|x_j|\leq |x_i|$ für alle 
$j\neq i$. 

Die $i$-te Komponente von $Ax$ ist dann gegeben durch 
%
\begin{align*}
  \lambda x_i 
  = (Ax)_i = \sum_{j=1}^{n} a_{jj}x_j = a_{ii}x_i + \sum_{j\neq i} a_{jj}x_j 
\end{align*}
%
Somit folgt 
%
\begin{align*}
  |\lambda-a_{ii}| 
  = \Big|\dfrac{1}{x_i}\sum_{j\neq i} a_{ij} x_j\Big|
  = \Big|\sum_{j\neq i} a_{ij} \dfrac{x_j}{x_i}\Big|
  \leq \sum_{j\neq i}|a_{ij}|
\end{align*}
%
Demnach liegt $\lambda\in K_i$.
\qed

\begin{colbox}{Korollar}
  Aus Satz \ref{satz:gerschgorin} folgt insbesondere für $m=1$, dass wenn ein Gerschgorin-Kreis keinen anderen schneidet,
  dass dieser genau einen Eigenwert enthält. Und wenn alle Gerschgorin-Kreise disjunkt sind, dann ist die Matrix 
  diagonalisierbar. 

  Weiter ist eine Matrix genau dann invertierbar, wenn keiner der Gerschgorin-Kreise die 0 enthält.
\end{colbox}

\begin{colbox}{Beispiel}
  Gegeben sei die Matrix 
  \begin{align*}A = \begin{pmatrix}
    1 & 0.1 & -0.2 \\ 0 & 2 & 0.4 \\ -0.2 & 0 & 3
  \end{pmatrix}\end{align*}
  Es ergeben sich die folgenden Gerschgorin-Kreise:
  \begin{align*}
    K_1 = \{z\in\C : |z-1|\leq 0.3\} \\
    K_2 = \{z\in\C : |z-2|\leq 0.4\} \\
    K_3 = \{z\in\C : |z-3|\leq 0.2\} \\
  \end{align*}
  \begin{center}
    \begin{tikzpicture}[scale=3]
    % Axes
    \draw[->] (-0.2,0) -- (4,0) node[right] {Re\((z)\)};
    \draw[->] (0,-1) -- (0,1) node[above] {Im\((z)\)};
    
    % Circle 1
    \draw[thick] (1,0) circle (0.3);
    \filldraw (1,0) circle (0.02) node[below] {\(1\)};
    \draw[thick] (1,0) -- ++(70:0.3);
    \node at ($(1,0)+(70:0.4)$) {\(r = 0.3\)};
    
    % Circle 2
    \draw[thick] (2,0) circle (0.4);
    \filldraw (2,0) circle (0.02) node[below] {\(2\)};
    \draw[thick] (2,0) -- ++(70:0.4);
    \node at ($(2,0)+(70:0.5)$) {\(r = 0.4\)};
    
    % Circle 3
    \draw[thick] (3,0) circle (0.2);
    \filldraw (3,0) circle (0.02) node[below] {\(3\)};
    \draw[thick] (3,0) -- ++(70:0.2);
    \node at ($(3,0)+(70:0.3)$) {\(r = 0.2\)};

    \end{tikzpicture}
  \end{center}
\end{colbox}

\subsubsection{Stabilität von Eigenwerten}

\begin{colbox}{Satz}[Stabilitätssatz]
  Sei $A\in\K^{n\times n}$ eine diagonalisierbare Matrix, d.h. es gibt eine invertierbare Matrix 
  $W=(W^{(1)}|\dots|W^{(2)})$ aus Eigenvektoren mit $A=W\Lambda W^{-1}$ und $\Lambda=\diag(\lambda_i(A))$.
  
  Für eine zweite Matrix $B\in\K^{n\times n}$ gibt es zu 
  jedem Eigenwert $\lambda(B)$ von $B$ einen Eigenwert $\lambda(A)$ von $A$, sodass
  %
  \begin{align*}
    |\lambda(A)-\lambda(B)|\leq\Cond_2(W)\cdot\vertn{2}{A-B}_2
  \end{align*}
  %
\end{colbox}

\textit{Beweis.} \\
Sei $\lambda=\lambda(B)$ kein Eigenwert von $A$, so gilt 
%
\begin{align*}
  \vertn{2}{(\lambda I - A)^{-1}}_2 
  &= \vertn{2}{(\lambda I - W\Lambda W^{-1})^{-1}}_2 \\
  &= \vertn{2}{W(\lambda I - \Lambda)^{-1}W^{-1}}_2 \\
  &\leq \vertn{2}{W}_2\cdot\vertn{2}{W^{-1}}_2\cdot\vertn{2}{(\lambda I-\Lambda)^{-1}}_2\\
  &= \Cond_2(W)\cdot \max_{i=1,\dots,n} |\lambda-\lambda_i(A)|^{-1}
\end{align*}
%
Mit dem Hilfssatz \ref{hsatz:normIneq} folgt dann die Behauptung. 
\qed 

Für hermitische Matrizen $A\in\K^{n\times n}$ existiert bekannterweise eine Orthonormalbasis des 
$\K^{n\times n}$ aus Eigenvektoren, sodass die Matrix $W$ als unitär angenommen werden kann, 
d.h. $W^{-1}=W^H$. In diesem Fall gilt $\vertn{2}{W} = \sqrt{\lambda_{\max}(W^HW)} = \sqrt{\lambda_{\max}(I)}=1$
und analog $\vertn{2}{W^H} = 1$, also 
%
\begin{align*}
  \Cond_2(W)=\vertn{2}{W^{-1}}_2\cdot\vertn{2}{W}_2 = 1
\end{align*}
%
\textbf{Regel:} Allgemein kann man sagen, dass das Eigenwertproblem für hermitische Matrizen gut konditioniert
ist, während das allgemeine Eigenwertproblem je nach Größe von $\Cond_2(W)$ beliebig schlecht konditioniert 
sein kann.

\subsection{Iterative Verfahren}

Im folgenden wollen wir ein iteratives Verfahren zu Lösung des partiellen Eigenwertproblems einer 
Matrix $A\in\K^{n\times n}$ betrachten.

\subsubsection{Potenz-Methode}

\begin{colbox}{Definition}
  Die Potenzmethode (Von-Mises-Iteration) erzeugt ausgehend von einem Startvektor $z^{(0)}\in\C^n$ mit
  $\vertn{2}{z^0}=1$ eine Folge von Iterationen $z^{(t)}\in\C^n, t=1,2,\dots$ durch 
  %
  \begin{align*}
    \tilde{z}^{(t)} 
    = Az^{(t-1)} 
    \quad\text{und}\quad 
    z^{(t)} 
    = \tfrac{\tilde{z}^{(t)}}{\vertn{2}{\tilde{z}^{(t)}}}
  \end{align*}
  Für einen beliebigen  Index $k\in\{1,\dots,n\}$, (z.B. maximale Komponente von $z^{(k)}$) setzen wir:
  %
  \begin{align*}
    \lambda^{(t)} 
    = \dfrac{(Az^{(t)})_k}{(z^{(t)})_k}
  \end{align*}
  %
  \textit{Zur Normierung wird üblicherweise $\vertn{2}{\cdot}=\vertn{2}{\cdot}_2$ oder $\vertn{2}{\cdot}_\infty$ verwendet.} 
\end{colbox}

Zur Analyse des Verfahrens nehmen wir an, dass die Matrix $A$ diagonalisierbar ist. 
Dies ist äquivalent zu der Tatsache, dass $A$ eine Basis von normierten Eigenvektoren 
$\{w^{(1)},\dots,w^{(n)}\}$ besitzt. 

zusätzlich nehmen wir an, dass $z^{(0)}$ eine nicht-triviale Komponente bezüglich $w^{(n)}$ besitzt. 
\textit{(Dies ist keine wesentliche Einschränkung, da aufgrund des unvermeidbaren Rundungsfehlers dieser Fall der 
Iteration sicher einmal auftritt)}

\begin{colbox}{Satz}[Potenz-Methode]
  Die Matrix $A$ sei diagonalisierbar und ihr betragsgrößter Eigenwert sei separiert von den anderen 
  Eigenwerten, also oBdA $|\lambda_n|>|\lambda_{n-1}|\geq|\lambda_{n-2}|\geq\dots\geq|\lambda_1|$. 

  Der Startvektor $z^{(0)}$ habe eine nicht-triviale Komponente bezüglich des zugehörigen Eigenvektors $w^{(n)}$,
  dann gibt es Zahlen $\delta_t\in\C, |\delta_t|=1$, 
  sodass 
  \begin{align*}
    \vertn{2}{z^{(t)}-\delta_t\cdot w^{(n)}}\rightarrow 0
    \quad \text{und}\quad 
    \lambda^{(t)}-\lambda_n = \mathcal{O}\left(\left|\dfrac{\lambda_{n-1}}{\lambda_n}\right|^t\right)
    \qquad 
    \text{ für } t\rightarrow \infty\end{align*}
\end{colbox}

\textit{Beweis.} \\
Sei $z^{(0)}=\displaystyle\sum_{i=1}^n \alpha_i\cdot w^{(i)}$ die Basisdarstellung des Startvektors 
(mit $\alpha_n\neq 0$). Für die Iterierten gilt:
%
\begin{align*}
  z^{(t)} 
  = \dfrac{\tilde{z}^{(t)}}{\vertn{2}{\tilde{z}^{(t)}}} 
  = \dfrac{Az^{(t-1)}}{\vertn{2}{Az^{(t-1)}}} 
  = \dots 
  = \dfrac{A^tz^{(0)}}{\vertn{2}{A^tz^{(0)}}}
  \tag{1}\label{eq:PowerEQ1}
\end{align*}
%
Dabei gilt:
%
\begin{align*}
  A^tz^{(0)} 
  = \sum_{i=1}^{n}\alpha_i\lambda_i^tw^{(i)} 
  = \lambda_n^t \alpha_n\cdot\left(w^{(n)} + 
  \sum_{i=1}^{n-1} \dfrac{\alpha_i}{\alpha_n}\left(\dfrac{\lambda_i}{\lambda_n}\right)^t w^{(i)}\right)
  \tag{2}\label{eq:PowerEQ2}
\end{align*}
%
Wegen $|\tfrac{\lambda_i}{\lambda_n}|\leq \rho:=|\tfrac{\lambda_{n-1}}{\lambda_n}|<1$ für $i=1,\dots,n-1$ folgt
\begin{align*}
  A^tz^{(0)} 
  = \lambda_n^t\alpha_n(w^{(n)}+\mathcal{O}(\rho^t))
  \qquad \text{ für } t\rightarrow \infty
\end{align*}
%
Dies ergibt nach (\ref{eq:PowerEQ1}):
%
\begin{align*}
  z^{(t)} 
  = \dfrac{A^tz^{(0)}}{\vertn{2}{A^tz^{(0)}}}
  = \dfrac{\lambda_n^t\alpha_n(w^{(n)}+\mathcal{O}(\rho^t))}
  {|\lambda_n^t\alpha_n|\cdot\vertn{2}{w^{(n)}+\mathcal{O}(\rho)}}
  = \underbrace{\dfrac{\lambda_n^t\alpha_n}{|\lambda_n^t\alpha_n|}}_{=:\delta_k}\cdot w^{(n)}+\mathcal{O}(\rho)
\end{align*}
%
Dabei ist $\delta_t\in\C$ und $|\delta_t|=1$, daher folgt die erste Aussage.

Weiter gilt 
%
\begin{align*}
  \lambda^{(t)} 
  &= \dfrac{(Az^{(t)})_k}{(z^{(t)})_k} \\
  &\stackrel{(\ref{eq:PowerEQ1})}{=} 
  \dfrac{(A^{t+1}z^{(0)})_k}{\vertn{2}{(A^{t+1}z^{(0)})_k}}\cdot\dfrac{{\vertn{2}{(A^{t+1}z^{(0)})_k}}}{(A^tz^{(0)})_k} \\
  &\stackrel{(\ref{eq:PowerEQ2})}{=} 
  \dfrac{\lambda_n^{t+1}\Big(\alpha_n w_{n,k} 
  + \displaystyle\sum_{i=1}^{n-1} \alpha_i(\tfrac{\lambda_i}{\lambda_n})^{t+1}w_{i,k}\Big)}
  {\lambda_n^t\Big(\alpha_n w_{n,k} 
  + \displaystyle\sum_{i=1}^{n-1} \alpha_i(\tfrac{\lambda_i}{\lambda_n})^{t}w_{i,k}\Big)} \\
  &= \lambda_n + \mathcal{O}\left(\left|\dfrac{\lambda_{n-1}}{\lambda_n}\right|^t\right)\qquad 
  \text{ für } t\rightarrow \infty
\end{align*}
%
\qed

Die Konvergenz der Potenzmethode ist umso besser, je mehr der betragsgrößte Eigenwert $\lambda_n$ von den übrigen 
separiert ist. Der Beweis ist verallgemeinerbar für betragsgrößte Eigenwerte, welche mehrfach 
auftreten, sofern die Matrix diagonalisierbar ist. 

\subsubsection{Inverse Iteration} 
Als nächstes wollen wir uns die \glqq{}Inverse Iteration\grqq{} nach
Wielandt anschauen. 

Wir nehmen an, wir haben bereits eine Näherung $\tilde{\lambda}$ für einen Eigenwert $\lambda_k$ der regulären Matrix 
$A$ (z.B. durch Einschließungssätze). Die Näherung sei gut in dem Sinne, 
dass $|\lambda_k-\tilde{\lambda}|\ll |\lambda_i-\tilde{\lambda}|$ für $i\neq k$.

Wir betrachten das Eigenwertproblem, welches sich für die Matrix $A-\tilde{\lambda}I$ ergibt:
%
\begin{align*}
  (A-\tilde{\lambda}I)v
  = \xi v 
  \iff 
  (A-\tilde{\lambda}I-\xi I)v 
  = 0 
  \iff (A-(\tilde{\lambda}+\xi)I)v
  = 0
\end{align*}
%
Wegen $(A-\lambda_kI)=0$, ist $\xi=\lambda_k-\tilde{\lambda}$ ein Eigenwert von $A-\tilde{\lambda}I$ und damit 
$\mu=\tfrac{1}{\xi}=(\lambda_k-\tilde{\lambda})^{-1}$ ein Eigenwert von $(A-\tilde{\lambda}I)^{-1}$.
\footnote{
    $Av=\lambda v \implies v = A^{-1}\lambda v \implies \lambda^{-1}v = A^{-1}v$
  } 

Allgemeiner hat im Falle $\tilde{\lambda}\neq\lambda_k$ die Matrix $(A-\tilde{\lambda}I)^{-1}$ die Eigenwerte 
$\mu_i = (\lambda_i-\tilde{\lambda})^{-1}$ für $i=1,\dots,n$ und es gilt 
%
\begin{align*}
  \left|\dfrac{1}{\lambda_k-\tilde{\lambda}}\right| 
  \gg \left|\dfrac{1}{\lambda_i-\tilde{\lambda}}\right|
  \qquad \text{für } i\neq k
\end{align*}

\begin{colbox}{Definition}
  Die inverse Iteration beruht auf der Anwendung der Potenzmethode auf die Matrix $(A-\tilde{\lambda}I)^{-1}$
  mit einer a priori Schätzung $\tilde{\lambda}$ zum gesuchten Eigenwert $\lambda_k$. 

  Ausgehend von einem Startwert $z^{(0)}$ werden Iterierte $z^{(t)}$ als Lösung der Gleichungssysteme
  %
  \begin{align*}
    (A-\tilde{\lambda}I)\tilde{z}^{(t)} 
    = \tilde{z}^{(t-1)},
    \qquad 
    z^{(t)} 
    = \dfrac{\tilde{z}^{(t)}}{\vertn{2}{\tilde{z}^{(t)}}}
  \end{align*}
  %
  bestimmt.

  Die zugehörige Eigenwertnäherung wird bestimmt durch 
  %
  \begin{align*}
    \mu^{(t)}=\dfrac{(z^{(t)})_k}{((A-\tilde{\lambda}I)z^{(t)})_k}
  \end{align*}
  %
  mit Nenner $\neq 0$ (oder im symmetrischen Fall einfach mit Hilfe der Rayleigh-Quotienten).
\end{colbox}

Aufgrund der Aussagen über Potenzmethoden liefert die inverse Iteration also für eine diagonalisierbare Matrix
jeden Eigenwert, zu dem bereits eine hinreichend gute Näherung bekannt ist.

\subsection{Page-Rank-Algorithmus}
Das Ziel des Page-Rank-Algorithmus ist die Bestimmung der Ausgabereihenfolge bei Suchergebnissen. Dabei berufen 
wir uns auf folgende Regeln:

\begin{enumerate}
  \item[(1)] Eine Website erhält eine umso höhere Bewertung, je mehr Links auf sie zeigen.
  \item[(2)] Links von höher bewerteten Websites soll relevanter sein, als solche von unbedeutenden Websites
  \item[(3)] Ein Link von einer Website, die wenig Links nach außen hat, soll höher gewichtet werden als der
    von einer Website mit vielen Links nach außen. 
\end{enumerate}

Wir beschreiben unser Model als ein Netz mit $n$ Websites, wobei ein Index $k$ immer für eine Website steht und suchen 
die Bewertung $x_k\in\R$ der Website $k$. 

Weiter sei $L_k$ die Menge der Websites, welche auf $k$ verlinken, 
Links auf Websites von sich selbst werden dabei nicht berücksichtigt und 
$n_k$ sei die Anzahl Websites, auf welche $k$ verlinkt.

Wir modellieren mittels folgendem LGS
%
\begin{align*}
  x_k = \sum_{j\in L_k}\tfrac{1}{n_j}\cdot x_j
\end{align*}
%
Die Gleichung $x=Ax$ mit 
\begin{align*}x = \begin{pmatrix}
  x_1 \\ \vdots \\ x_n
\end{pmatrix}, \qquad A_{ij} = a_{ij} = \begin{cases}
  \tfrac{1}{n_j}, & \text{falls die Seite } j \text{ auf die Seite } i \text{ verlinkt} \\
  0, & \text{sonst}
\end{cases}\end{align*}
entspricht hierbei der Eigenwertgleichung für den Eigenwert $\lambda=1$. \\

Der historische Ansatz von Google ist die Potenzmethode:

\subsubsection{Stochastische Vektoren/Matrizen}
\begin{colbox}{Definition}
  Ein Vektor $p\in\R^n$ heißt stochastischer Vektor, wenn alle Elemente $p_i$ nicht-negativ sind und die 
  Summe der Elemente des Vektors gleich 1 ist, d.h. $\sum_{i} p_i = 1$. 

  Eine Matrix $A\in\R^{n\times n}$ heißt stochastische Matrix, wenn alle Spalten der Matrix stochastische 
  Vektoren sind, d.h.
  %
  \begin{align*}
    a_{ij}\geq 0\ \forall i,j \quad \text{und } \quad \sum_{i=1}^n a_{ij}=1\ \forall j
  \end{align*}

  %Wir schreiben im folgenden auch $\stochM{n}$ und $\stochV{n}$ für die Menge der stochastischen Matrizen und Vektoren.
\end{colbox}

\begin{colbox}{Lemma}
  Sei $A\in\R^{n\times n}$ eine stochastische Matrix und $p\in\R^{n}$ ein stochastischer Vektor, dann ist 
  das Produkt $Ap\in\R^{n\times n}$ wieder ein stochastischer Vektor.
\end{colbox}

\textit{Beweis.} \\
Offensichtlich $(Ap)_{ij}\geq 0$, weiter es sei $a_i$ die $i$-te Spalte der Matrix $A$, d.h. $a_i$ ist ein stochastischer Vektor
\begin{align*}
  A\cdot p = A\cdot \begin{pmatrix}
  p_1 \\ \vdots \\ p_n
\end{pmatrix} &= p_1\cdot a_1 + \dots + p_n\cdot a_n \\
&= \sum_{i=1}^n (p_1a_{i1} + \dots + p_na_{in}) \\
&= p_1\sum_{i=1}^n a_{i1} + \dots +  p_n\sum_{i=1}^na_{in} \\
&= p_1\cdot 1 + \dots + p_n\cdot 1 = 1
\end{align*}
\qed

\begin{colbox}{Korollar}\label{cor:stochMatProd}
  Seien $A,B\in\R^{n\times n}$ stoch. Matrizen, dann ist das Produkt $A\cdot B$ wieder eine stoch. Matrix.
\end{colbox}

\textit{Beweis.} Folgt direkt aus Lemma 3.9.

\begin{colbox}{Satz}
  Eine stochastische Matrix $A$ hat immer den Eigenwert $1$. Der Betrag aller anderen Eigenwerte ist
  kleiner oder gleich 1.
\end{colbox}
\textit{Beweis.} \\
Für den ersten Teil nutzen wir aus, dass $A$ und $A^T$ die gleichen Eigenwerte, da $A$ und $A^T$ die 
gleiche Determinante besitzen und damit die charakteristischen Polynome $\chi_A(\lambda) = \det(A-\lambda I) 
= \det(A^T-\lambda I) = \chi_{A^T}(\lambda)$ übereinstimmen. 

Weiter ist nach Definition einer stochastischen Matrix die Summe jedes Zeilenvektors von $A^T$ ist gleich 1,
also ist $e=(1,\dots,1)^T$ ein Eigenvektor von $A^T$ mit Eigenwert $1$, also besitzt auch die Matrix $A$ den 
Eigenwert $\lambda = 1$. 

Für den zweiten Teil nehmen wir an es existiert ein Eigenvektor $v$ zum Eigenwert $\lambda$ mit $|\lambda|>1$, 
denn dann gilt 
%
\begin{align*}
  A^n v = A^{n-1}(Av) = A^{n-1}\lambda v = \lambda A^{n-1}v = \dots = \lambda^n v
\end{align*}
%
Für die Länge dieses Vektors gilt $\vertn{2}{\lambda^n v} = |\lambda^n|\cdot \vertn{2}{v}$ ein 
exponentielles Wachstum in $n$, da $|\lambda|>1$. 

Daraus folgt, dass für große $n$ ein Element $(A^n)_{ij}$ existiert, welches größer als $1$ ist.

Da nach Korollar \ref{cor:stochMatProd} die Matrix $A^n$ stochastisch ist bildet dies einen Widerspruch.
\qed

\begin{colbox}{Lemma}
  Die Bewertungsmatrix $A$ des Page-Rank-Algorithmus ist eine stochastische Matrix.
\end{colbox}
\textit{Beweis.} Offensichtlich gilt $a_{ij}\geq 0$, weiter gilt
\begin{align*}\sum_{i=1}^n a_{ij} = n_j\cdot \dfrac{1}{n_j} + (n-n_j)\cdot 0 = 1\end{align*}
\qed
\begin{egbox}
  Wir betrachten folgendes einfaches Netz mit $4$ Knoten: 

  \begin{center}
    \begin{tikzpicture}[>=Stealth, node distance=2.5cm, every node/.style={circle, draw, minimum size=1cm}, auto]

% Nodes
\node (1) at (0,2.5) {1};
\node (2) at (0,0) {2};
\node (3) at (2.5,2.5) {3};
\node (4) at (2.5,0) {4};

% Edges
\draw[->, bend left=15] (1) to (3);
\draw[->, bend left=15] (3) to (1); 
\draw[->, bend left=15] (1) to (4); 
\draw[->, bend left=15] (4) to (1); 
\draw[->] (1) -- (2);
\draw[->] (2) -- (4);
\draw[->] (4) -- (3);
\draw[->] (2) -- (3);

\end{tikzpicture}
  \end{center}

  Es ergibt sich folgendes Gleichungssystem:
  %
  \begin{align*}
    \begin{pmatrix}
      x_1\\ x_2\\ x_3\\ x_4
    \end{pmatrix} 
    = \begin{pmatrix}
      0 & 0 & 1 & 1/2 \\ 
      1/3 & 0 & 0 & 0 \\
      1/3 & 1/2 & 0 & 1/2 \\
      1/3 & 1/2 & 0 & 0
    \end{pmatrix}\cdot 
    \begin{pmatrix}
      x_1\\ x_2\\ x_3\\ x_4
    \end{pmatrix}
  \end{align*}
  %
  Lösen dieses linearen Gleichungssystems liefert:
  %
  \begin{align*}
    x \in \text{span}\left\{\begin{pmatrix}
    0.72 \\ 0.24 \\ 0.54 \\ 0.36
    \end{pmatrix}\right\}
  \end{align*}
  %
  Demnach hat die erste Website die höchste Bewertung. 
\end{egbox}

\subsubsection{Vorgehensweite für weitere Eigenwerte/Eigenvektoren}
Wir betrachten die Diagonalmatrix
%
\begin{align*}
  A = \begin{pmatrix}
    3 & 0 \\ 0 & 2
  \end{pmatrix}
\end{align*}
%
mit den Eigenwerten $\lambda_1=3$ und $\lambda_2=2$ zu den Eigenvektoren
%
\begin{align*}
  w^{(1)}=\begin{pmatrix}
    0 \\ 1
  \end{pmatrix}, \quad w^{(2)}= \begin{pmatrix}
    1 \\ 0
  \end{pmatrix}
\end{align*}
%
Wir führen folgende Transformation durch:
%
\begin{align*}
  B = \underbrace{\begin{pmatrix}
    3 & 0 \\ 0 & 2
  \end{pmatrix}}_{=A} - \underbrace{\begin{pmatrix}
    3 & 0 \\ 0 & 0
  \end{pmatrix}}_{=(1,0)^T(3,0)} 
  = \begin{pmatrix}
    0 & 0 \\ 0 & 2
  \end{pmatrix}
\end{align*}

\textbf{Idee:} Umwandlung der betrachteten Matrix in eine andere Matrix, wobei der betragsgrößte Eigenwert entfernt
wird, d.h. durch $0$ ersetzt wird. Eine Iterative Anwendung liefert dann alle Eigenwerte. 

Sei $A\in\R^{n\times n}$ gegeben mit betragsmäßig absteigenden Eigenwerten, d.h. 
%
\begin{align*}
  |\lambda_1| > |\lambda_2| > \dots > 0
\end{align*}
%
Der Eigenvektor zu $\lambda_1$ sei gegeben durch $u^{(1)}$. \\
Wir wählen eine von Null verschiedene Komponente $u^{(1)}_p$ von $u^{(1)}$ und schreiben $A_p$ für die $p$-te Zeile
von $A$, d.h. $A_p=(A_{p1},A_{p2},\dots,A_{pn})^T$. Betrachte nun die Matrix 
%
\begin{align*}
  B 
  = A - \dfrac{1}{w_p^{(1)}} w^{(1)}\cdot A_p^T\quad\text{ mit }\quad B_{ij} = 
  A_{ij}-\dfrac{1}{w^{(1)}_p}w^{(1)}_i\underbrace{A_{pj}}_{A_{pj}}
\end{align*}
%
Aus dem Eigenwertproblem $Aw^{(k)} = \lambda_k w^{(k)}$ ergibt sich
%
\begin{align*}
  \lambda^{(k)}w^{(k)}_p = (Aw^{(k)})_p = \langle A_p, w^{(k)}\rangle
\end{align*}
%
Für $k=1$ ergibt sich:
%
\begin{align*}
  Bw^{(1)} 
  &= Aw^{(1)} - \dfrac{1}{w^{(1)}_p} \cdot w^{(1)}\cdot \langle A_p, w^{(1)}\rangle \\
  &= Aw^{(1)} - \dfrac{1}{w^{(1)}_p} \cdot w^{(1)}\cdot \lambda_1 w^{(1)}_p \\
  &= \lambda_1 w^{(1)} - \lambda_1w^{(1)} \\
  &= 0
\end{align*}
%
d.h. $0$ ist ein Eigenwert von $B$  (statt vorher $\lambda^{(1)}$ von $A$).  

Analoge Überlegung für $k=2,\dots,n$ liefert:
%
\begin{align*}
  Bw^{(k)} 
  &= \lambda_k w^{(k)} - \dfrac{1}{w^{(1)}_p} \cdot w^{(1)}\cdot \lambda_kw^{(k)}_p \\
  &= \lambda_k\cdot\left(w^{(k)} - \dfrac{w^{(k)}_p}{w^{(1)}_p} \cdot w^{(1)}\right) 
  \tag{1}\label{eq:WielandtEQ1}
\end{align*}
%
Die Eigenwerte bleiben beim Wechsel von $A$ zu $B$ erhalten, da 
\begin{align*}
  Bw^{(k)} + 0 
  &= Bw^{(k)} + \underbrace{Bw^{(1)}}_{=0} \\
  &= Bw^{(k)} + Bw^{(1)}\cdot\dfrac{-w_p^{k}}{w_p^{(1)}} \\
  &= B\cdot\left(w^{(k)} - \dfrac{w^{(k)}_p}{w^{(k)}_p} \cdot w^{(k)}\right) 
  \stackrel{(\ref{eq:WielandtEQ1})}{=} 
  \lambda_k\cdot\left(w^{(k)} - \dfrac{w^{(k)}_p}{w^{(k)}_p} \cdot w^{(k)}\right)
  \tag{2}\label{eq:WielandtEQ2}
\end{align*}
Die Gleichung (\ref{eq:WielandtEQ2}) zeigt für $k=2,3,\dots,n$, dass $\lambda_k$ auch ein Eigenwert zu $B$ ist, 
wenn auch mit anderem Eigenvektor. 

\begin{colbox}{Satz}[Deflation nach Wielandt]
  Seien die Eigenwerte von $A$, betragsmäßig absteigend, d.h. 
  $|\lambda_1|>|\lambda_2|>\dots>|\lambda_n|$, mit zugehörigen Eigenvektoren $u^{(1)},\dots,u^{(n)}$.
  Dann besitzt die Matrix 
  %
  \begin{align*}
    B 
    = A - \dfrac{1}{u_p^{(1)}} u^{(1)}\cdot a^T
    \quad\text{ mit }\quad u_p^{(1)}
    \neq 0 
    \text{ und } a^T=(A_{p1},A_{p2},\dots,A_{pn})
  \end{align*}
  die Eigenwerte $0,\lambda_2,\dots,\lambda_n$ mit den zugehörigen Eigenvektoren 
  $w^{(1)},\tilde{w}^{(2)},\dots,\tilde{w}^{(n)}$, wobei
  % 
  \begin{align*}
    w^{(k)} 
    = u^{(k)} - \dfrac{u_p^{(k)}}{u_p^{(1)}}\cdot u^{(1)}\tag{*}
  \end{align*}
  % 
  Den betragsmäßig zweiten Eigenwert $\lambda_2$ zugehörigen Eigenvektor $w^{(2)}$ erhält man somit mit der 
  Potenzmethode für die Matrix $B$ nach ihrer Definition. \\ 
  Der Eigenvektor $u^{(2)}$ zum Eigenwert $\lambda_2$ der Matrix $A$ kann wie folgt rekonstruiert werden:
  \begin{enumerate}
    \item[a)] Lösen des linearen Gleichungssystems (*) bezüglich $u^{(2)}$ 
    \item[b)] Lösen des LGS der EW-Gleichung  
    \item[c)] Inverse Iteration nach Wielandt anwenden, um Eigenvektor von $A$ zum zugehörigen Eigenwert
    $\lambda_2$ zu erhalten 
  \end{enumerate}
\end{colbox}

\begin{egbox}
  Gesucht seien die Eigenwerte und Eigenvektoren der Matrix
  \begin{align*}A = \begin{pmatrix}
    -4 & 14 & 0 \\ -5 & 13 & 0 \\ -1 & 0 & 2
  \end{pmatrix}\end{align*}
  Im ersten Schritt verwenden wir die Potenzmethode um den betragsmäßig größten Eigenwert und den zugehörigen
  Eigenvektor zu bestimmen, wir erhalten:
  \begin{align*}
  \lambda_1=6 \quad, u^{(1)}=\begin{pmatrix}
      -4 \\ \nicefrac{-20}{7} \\ 1
    \end{pmatrix}
  \end{align*}
  Für die Deflation wählen wir nun $p=1$ mit $u_p^{(1)}\neq 0$ und $a^T=(-4, -5, -1)$, die resultierende Matrix $B$
  ergibt sich dann durch:
  \begin{align*}
  B = \begin{pmatrix}
    -4 & 14 & 0 \\ -5 & 13 & 0 \\ -1 & 0 & 2
  \end{pmatrix} - \dfrac{1}{-4}\cdot\begin{pmatrix}
    -4 \\ -\tfrac{20}{7} \\ 1
  \end{pmatrix}\cdot(-4, 14, 0) = \begin{pmatrix}
    0 & 0 & 0 \\ \nicefrac{-15}{7} & 3 & 0 \\ -2 & \nicefrac{7}{2} & 2
  \end{pmatrix}
  \end{align*}
  Erneutes Anwenden der Potenzmethode auf die neue Matrix $B$ liefert den zweigrößten Eigenwert und den
  zugehörigen Eigenvektor von $B$:
  \begin{align*}
    \lambda_2=3 \quad, w^{(2)}\begin{pmatrix}
      0 \\ \nicefrac{2}{7} \\ 1
    \end{pmatrix}
  \end{align*}
  Eine weitere Deflation mit dem neu gewonnen Eigenvektor und $p=3$ ergibt
  \begin{align*}
    C = \begin{pmatrix}
    0 & 0 & 0 \\ \nicefrac{-15}{7} & 3 & 0 \\ -2 & \nicefrac{7}{2} & 2
  \end{pmatrix} - \dfrac{1}{1}\cdot \begin{pmatrix}
      0 \\ \nicefrac{2}{7} \\ 1
    \end{pmatrix} \cdot \begin{pmatrix}
      -2 & \nicefrac{7}{2} & 2
    \end{pmatrix} = \left(\begin{matrix}
      0 & 0 & 0 \\
      \nicefrac{-11}{7} & 2 & \nicefrac{-4}{7} \\
      0 & 0 & 0
    \end{matrix}\right)
  \end{align*}
  Hierbei ergibt sich der letzte Eigenwert und der zu $C$ zugehörige Eigenvektor
  \begin{align*}
    \lambda_3=2 \quad, v^{(3)}=\begin{pmatrix}
      0 \\ 1 \\ 0
    \end{pmatrix}
  \end{align*}
  Nach beliebiger Methodik aus Satz 3.14 lassen sich nun aus $w^{(2)}$ und $v^{(3)}$ die Eigenvektoren  $u^{(2)}$
  und $u^{(3)}$ von $A$ konstruieren.
\end{egbox}