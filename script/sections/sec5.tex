% !TeX root = ../script.tex

\section{Die schnelle Fourier-Transformation}

Im folgenden Abschnitt wollen wir uns mit der schnellen Fourier-Transformation (\glqq{FFT} - fast Fourier transform) 
als zentrales Werkzeug der Signalverarbeitung und Bildkompression. 
Um die Idee hinter dem FFT-Algorithmus zu verstehen beginnen wir mit einer kurzen Wiederholung zu Fourier-Reihen.

\subsection{Fourier-Reihen}
Wir betrachten $f$ eine $2\pi$-periodische Funktion (d.h. $f(x+2\pi)=f(x)$ für alle $x\in\R$) mit 
dem Ziel eine Annäherung durch Linearkombinationen $2\pi$-periodischen Funktionen $\{\cos(kx)\}_{k=0}^{n}$ und 
$\{\sin(kx)\}_{k=1}^{n}$ zu finden:
%
\begin{align*}
  g_n(x) = \dfrac{1}{2}a_0 + \sum_{k=1}^{n}\Big(a_k\cos(kx)+b_k\sin(kx)\Big)
\end{align*}
%
Wir suchen eine Approximation im Sinne der $L_2$ Norm, d.h. wir minimieren den Ausdruck
%
\begin{align*}
  \vertn{2}{g_n(x)-f(x)}_2 = \left(\int_{0}^{2\pi} (g_n(x)-f(x))^2 \diff x\right)^{1/2}
\end{align*}

\begin{colbox}{Satz}[Fourier-Koeffizienten]
  Für trigonometrisches Polynom, d.h. eine Funktion der Form  
  %
  \begin{align*}
    g_n(x) = \dfrac{1}{2}a_0 + \sum_{k=1}^{n}\Big(a_k\cos(kx)+b_k\sin(kx)\Big)
  \end{align*}
  %
  gilt 
  %
  \begin{align*}
    a_k 
    &= \dfrac{1}{\pi} \int_{0}^{2\pi} g_n(x)\cos(kx) \diff x, \quad k=0,1,\dots,n \\
    b_k 
    &= \dfrac{1}{\pi} \int_{0}^{2\pi} g_n(x)\sin(kx) \diff x, \quad k=1,\dots,n 
  \end{align*}
  %
\end{colbox}

\textit{Beweis.} Durch Verwendung der Orthogonalitätsbedingungen der trigonometrischen Funktionen
%
% \begin{align*}
%   \int_{0}^{2\pi} \cos(kx)\cos(lx) \diff x 
%   &= \begin{cases}
%     \pi, & \text{falls } k = l \neq 0, \\
%     2\pi, & \text{falls } k = l = 0, \\
%     0, & \text{falls } k \neq l,
%   \end{cases}
%   \\
%   \int_{0}^{2\pi} \sin(kx)\sin(lx) \diff x 
%   &= \begin{cases}
%   \pi, & \text{falls } k = l, \\
%   0, & \text{falls } k \neq l,
%   \end{cases}
%   \\
%   \int_{0}^{2\pi} \cos(kx)\sin(lx)\diff x &= 0
% \end{align*}
% %
ergibt sich für $l=0$:
%
\begin{align*}
  \dfrac{1}{\pi} \int_{0}^{2\pi} g_n(x)\underbrace{\cos(lx)}_{1} \diff x 
  &= \dfrac{1}{\pi} \int_{0}^{2\pi} \left(\dfrac{1}{2}a_0 + \sum_{k=1}^{n}\Big(a_k\cos(kx)+b_k\sin(kx)\Big)\right)\cdot 1 \diff x \\
  &= \dfrac{1}{2\pi} a_0\cdot \int_{0}^{2\pi} 1 \diff x
  + \sum_{k=1}^n \left(
    \dfrac{1}{\pi} a_k \cdot\underbrace{\int_{0}^{2\pi} \cos(kx) \diff x}_{0} 
    + \dfrac{1}{\pi} b_k \cdot \underbrace{\int_{0}^{2\pi} \sin(kx)\diff x}_{0}
  \right) \\
  &= \dfrac{1}{2\pi} a_0 \cdot 2\pi = a_0
\end{align*}
%
und für $1\leq l \leq n$:
\begin{align*}
  &\dfrac{1}{\pi} \int_{0}^{2\pi} g_n(x)\cos(lx) \diff x \\
  &\qquad= \dfrac{1}{2\pi} a_0 \cdot\underbrace{\int_{0}^{2\pi} \cos(lx) \diff x}_{0}
  + \sum_{k=1}^n \left(
    \dfrac{1}{\pi}a_k\cdot\underbrace{\int_{0}^{2\pi} \cos(kx)\cos(lx) \diff x}_{\pi\text{ wenn } l=k, \text{ sonst } 0}
    + \dfrac{1}{\pi} b_k\cdot\underbrace{\int_{0}^{2\pi} \sin(kx)\cos(lx)\diff x}_{0}
  \right) \\
  &\qquad= \dfrac{1}{\pi} a_l \cdot \pi = a_l
\end{align*}
und 
\begin{align*}
  &\dfrac{1}{\pi} \int_{0}^{2\pi} g_n(x)\sin(lx) \diff x \\
  &\qquad= \dfrac{1}{2\pi} a_0\cdot \underbrace{\int_{0}^{2\pi} \sin(lx) \diff x}_{0}
  + \sum_{k=1}^n \left(
    \dfrac{1}{\pi}a_k\cdot\underbrace{\int_{0}^{2\pi} \cos(kx)\sin(lx) \diff x}_{0}
    + \dfrac{1}{\pi} b_k\cdot\underbrace{\int_{0}^{2\pi} \sin(kx)\sin(lx)\diff x}_{\pi\text{ wenn } l=k, \text{ sonst } 0}
  \right) \\
  &\qquad= \dfrac{1}{\pi} b_l \cdot \pi = b_l
\end{align*}
\qed
%

In Unserem Fall, wo wir $f$ durch $g_n$ annähern wollen verwenden wir daher $f(x)$ bei der Bestimmung unserer 
Koeffizienten.

Im Allgemeinen ergeben sich für die Fourier-Koeffizienten $\{a_k\}_{k=0}^n$ und $\{b_k\}_{k=1}^n$ keine geschlossenen 
Formeln, d.h. wir sind auf numerische Integration angewiesen um diese zu bestimmen.

Verenden wir die Trapezregel als Quadraturformel um diese numerische Integration durchzuführen:

\begin{colbox}{Definition}[Trapezregel]
  Ein Verfahren zur numerischen Integration einer Funktion $f:[a,b]\to\R$ wird durch die Trapezregel beschrieben. Sie 
  beruht auf der Idee das Intervall $[a,b]$ in kleinere Intervalle $[x_j, x_{j+1}]$ für $j=0,\dots,N-1$ mit 
  $a=x_0<x_1<\dots<x_N=b$ aufzuteilen und die Funktion auf jedem dieser Intervalle als linear anzunehmen, 
  dies ermöglicht folgende Annäherung 
  %
  \begin{align*}
    \int_{x_j}^{x_{j+1}} f(x) \diff x \approx (x_{j+1}-x_j)\cdot\dfrac{f(x_{j+1})+f(x_j)}{2}
  \end{align*}
  %
  Insbesondere für den Fall von äquidistant gewählten Stützstellen mit Schrittweite $h=\tfrac{b-a}{N}$ ergibt sich
  %
  \begin{align*}
    \int_{a}^{b} f(x) \diff x \approx \dfrac{h}{2}\left(f(a) + 2\cdot\sum_{j=1}^{N-1}f(a + h\cdot j) + f(b) \right)
  \end{align*}
  %
\end{colbox}

Verwenden wir diese Trapezregel mit $x_j=\tfrac{2\pi}{N}\cdot j$ um unsere Fourier-Koeffizienten anzunähern 
ergibt sich die diskrete Fourier-Transformation (DFT):
%
\begin{align*}
  a_k &\approx\dfrac{1}{N}\left(f(x_0)\cdot\cos(kx_0) 
  + 2\sum_{j=1}^{N-1} f(x_j)\cdot\cos(kx_j) + f(x_N)\cdot \cos(kx_N)\right) \\
  b_k &\approx\dfrac{1}{N}\left(f(x_0)\cdot\sin(kx_0)  
  + 2\sum_{j=1}^{N-1} f(x_j)\cdot\sin(kx_j) + f(x_N)\cdot \sin(kx_N)\right)
\end{align*}
%
Mit Berücksichtigung der $2\pi$-Periodizität von $f$ ergeben sich für $a_k$ und $b_k$ die Näherungswerte
%
\begin{align*}
  a_k^* &:= \dfrac{2}{N}\sum_{j=1}^N f(x_j)\cdot \cos(kx_j), \quad k=0,1,2,\dots \\
  b_k^* &:= \dfrac{2}{N}\sum_{j=1}^N f(x_j)\cdot \cos(kx_j), \quad k=1,2,3\dots
\end{align*}

\textcolor{red}{Intuition Bilder}

\begin{colbox}{Lemma}\label{lem:diskStstelSum}
  Für die diskreten Stützstellen $x_j=\tfrac{2\pi}{N}\cdot j$ mit $1\leq N$  gilt
  %
  \begin{align*}
    \sum_{j=1}^N \cos(kx_j) 
    &= \begin{cases}
      0, & \text{falls } \tfrac{k}{N}\notin\Z \\
      N, & \text{falls } \tfrac{k}{N}\in\Z
    \end{cases} \\
    \sum_{j=1}^N \sin(kx_j) 
    &= 0 \text{ für alle } k\in\Z
  \end{align*}
  %
\end{colbox}

\textit{Beweis.}\\
Wir betrachten die komplexe Kombination beider Ausdrücke und erhalten
%
\begin{align*}
  S_N := \sum_{j=1}^{N} \cos(kx_j) + i\sin(kx_j) = \sum_{j=1}^{N} e^{ikx_j} = \sum_{j=1}^{N} e^{ik\cdot jh}
\end{align*}%
Dies ist eine endliche geometrische Reihe mit komplexem $q := e^{ikh} = e^{2\pi ik/N}$

Ist $\tfrac{k}{N}\notin\Z$, dann ist $q\neq 1$, und die Summenformel der endlichen geometrischen Reihe liefert 
%
\begin{align*}
   S_N 
   = e^{ikh}\dfrac{e^{ikhN}-1}{e^{ikh}-1} 
   = e^{ikh}\cdot\dfrac{e^{2\pi ki}-1}{e^{ikh}-1} 
   = 0, \text{wenn } \dfrac{k}{N}\notin\Z
\end{align*}
%
Für $\tfrac{k}{N}\in\Z$ folgt wegen $q=1$, dass $S=N$ ist. 

Die Unabhängigkeit von Real- und Imaginärteil schließt den Beweis.
\qed

\begin{colbox}{Satz}
  Die trigonometrischen Funktionen erfüllen für die äquidistanten Stützstellen $x_j$ 
  die diskreten Orthogonalitätsrelationen:
  %
  \begin{align*}
    \sum_{j=1}^{N} \cos(kx_j)\cos(lx_j) = \begin{cases}
      0, &\text{falls } \tfrac{k+l}{N}\notin \Z \text{ und } \tfrac{k-l}{N}\notin \Z \\
      N &\text{falls } \tfrac{k+l}{N}\in \Z \text{ und } \tfrac{k-l}{N}\in \Z \\
      \tfrac{N}{2} &\text{falls } \tfrac{k+l}{N}\in \Z \text{ und } \tfrac{k-l}{N}\notin \Z \\
      \tfrac{N}{2} &\text{falls } \tfrac{k+l}{N}\notin \Z \text{ und } \tfrac{k-l}{N}\in \Z
    \end{cases}
    % = \begin{cases}
    %   0, &\text{falls } \tfrac{k+l}{N}\notin \Z \text{ und } \tfrac{k-l}{N}\notin \Z \\
    %   N &\text{falls } \tfrac{k+l}{N}\in \Z \text{ und } \tfrac{k-l}{N}\in \Z\\
    %   \tfrac{N}{2} &\text{falls entweder } \tfrac{k+l}{N}\in \Z \text{ oder } \tfrac{k-l}{N}\in \Z \\
    % \end{cases}
  \end{align*}
  %
  und 
  %
  \begin{align*}
    \sum_{j=1}^{N} \sin(kx_j)\sin(lx_j) = \begin{cases}
      0, &\text{falls } \tfrac{k+l}{N}\notin \Z \text{ und } \tfrac{k-l}{N}\notin \Z \\
      0 &\text{falls } \tfrac{k+l}{N}\in \Z \text{ und } \tfrac{k-l}{N}\in \Z \\
      -\tfrac{N}{2} &\text{falls } \tfrac{k+l}{N}\in \Z \text{ und } \tfrac{k-l}{N}\notin \Z \\
      \tfrac{N}{2} &\text{falls } \tfrac{k+l}{N}\notin \Z \text{ und } \tfrac{k-l}{N}\in \Z
    \end{cases}
  \end{align*}
  %
  und 
  %
  \begin{align*}
    \sum_{j=1}^{N} \cos(kx_j)\sin(lx_j) = 0 \quad \text{für all } k,l\in\N
  \end{align*}
  %
\end{colbox}

\textit{Beweis.} \\
Zur Überprüfung der Orthogonalitätsrelationen werden die trigonometrischen Identitäten
\begin{align*}
  \cos(kx_j)\cos(lx_j) = \tfrac{1}{2}\Big(\cos(\big(k+l\big)x_j) + \cos(\big(k-l\big)x_j)\Big) \\
  \sin(kx_j)\sin(lx_j) = \tfrac{1}{2}\Big(\cos(\big(k-l\big)x_j) - \cos(\big(k+l\big)x_j)\Big) \\
  \cos(kx_j)\sin(lx_j) = \tfrac{1}{2}\Big(\sin(\big(k+l\big)x_j) - \sin(\big(k-l\big)x_j)\Big) \\
\end{align*}
verwendet und das Lemma \ref{lem:diskStstelSum} angewandt.
\qed 

\begin{colbox}{Satz}
  Sei $N=2n$ mit $n\in\N$. Das Fourier-Polynom
  %
  \begin{align*}
  g_m^*(x) := \tfrac{1}{2}a^*_0 + \sum_{k=1}^{m}\Big(a^*_k\cos(kx)+b^*_k\sin(kx)\Big)
  \end{align*}
  %
  von Grad $m<n$ mit Koeffizienten $a_k^*$ und $b^*_k$ approximiert die Funktion $f(x)$ im diskreten quadratischen 
  Mittel der $N$ Stützstellen $x_j$ derart, dass die Summe der quadratischen Abweichungen 
  %
  \begin{align*}
    F:=\sum_{j=1}^{N}\Big(g^*_m(x_j)-f(x_k)\Big)^2
  \end{align*}
  %
  minimal ist.
\end{colbox}
\textit{ohne Beweis.}


\begin{colbox}{Beispiel}
  Sei $f(x)=x^2$: \\
  \textcolor{red}{$x^2$ Plot} 
\end{colbox}

\subsection{Effiziente Berechnung der Fourier-Koeffizienten}
Die näherungsweise Berechnung der Fourier-Koeffizienten $a_k^*$ und $b_k^*$ ist für eine große Anzahl $N$ der 
Stützstellen sehr aufwendig. 

Dies ist vor allem bei der diskreten Fourier-Transformation relevant, die in Ingenieur- und Naturwissenschaften 
häufig eingesetzt wird, um z.B. die Frequenzen von Vibrationen zu bestimmen. 

Zur Berechnung der Summen 

%
\begin{align*}
  a'_k := \sum_{j=0}^{N-1} f(x_j)\cos(kx_j), \quad k=0,1,\dots,\tfrac{N}{2} \\
  b'_k := \sum_{j=0}^{N-1} f(x_j)\sin(kx_j), \quad k=1,2,\dots,\tfrac{N}{2}-1 
  \tag{1}\label{eq:fftEQ1}
\end{align*}
%

werden normalerweise $\propto N^2$ trigonometrische Funktionsauswertungen verlangt. Für den Fall, dass $N$ eine 
Potenz von 2 ist, kann ein sehr effizienter Algorithmus ($\propto n\log(n)$ Auswertungen) entwickelt werden, 
indem wir zu einer komplexen Fourier-Transformation übergehen.

%Aus zwei aufeinanderfolgenden Stützwerten bildet man die $n=N\,/\,2$ komplexen Zahlenwert:
%\begin{align*}y_j := f(x_{2j}) + i\cdot f(x_{2j+1}),\qquad \text{für } j=0,1,\dots,n-1\end{align*}

\begin{colbox}{Definition}[Diskrete komplexe Fourier-Transformation]
  Für eine Folge von komplexen Zahlen $f=(f_0,\dots,f_{n-1})^T\in\C^n$ ergibt sich die diskrete 
  komplexe Fourier-Transformation $\hat{f}$ durch
  %
  \begin{align*}
    \hat{f}_k := \sum_{j=0}^{n-1} f_j\cdot e^{-2\pi i\cdot\frac{jk}{n}} = \sum_{j=0}^{n-1} f_j\cdot \omega_n^{jk}
  \end{align*}
  %
  Dabei sind $\omega_n$ die $n$-ten Einheitswurzeln:
  %
  \begin{align*}
    \omega_n := e^{-2\pi i \,/\,n} = \cos\left(\dfrac{2\pi}{n}\right)+i\cdot\sin\left(\dfrac{2\pi}{n}\right)
  \end{align*}
  %
  In Matrix Schreibweise entspricht dies 
  %
  \begin{align*}
    \begin{pmatrix}
      \hat{f}_0 \\
      \hat{f}_1 \\
      \hat{f}_2 \\
      \vdots \\
      \hat{f}_{n-1}
    \end{pmatrix}
    =
    \begin{pmatrix}
      1 & 1 & 1 & \cdots & 1 \\
      1 & \omega_n & \omega_n^2 & \cdots & \omega_n^{n-1} \\
      1 & \omega_n^2 & \omega_n^4 & \cdots & \omega_n^{2(n-1)} \\
      \vdots & \vdots & \vdots & \ddots & \vdots \\
      1 & \omega_n^{n-1} & \omega_n^{2(n-1)} & \cdots & \omega_n^{(n-1)^2}
    \end{pmatrix}
    \cdot
    \begin{pmatrix}
      f_0 \\
      f_1 \\
      f_2 \\
      \vdots \\
      f_{n-1}
    \end{pmatrix}
  \end{align*}
  %
\end{colbox}

In unserem Fall haben wir eine reellwertige Funktion $f$ und damit auch reellwertige Punkte $f(x_j)$, 
wir können dennoch eine diskrete komplexe Fourier-Transformation durchführen und uns aus dem Resultat 
dann unsere reelle diskrete Fourier-Transformation berechnen:

\begin{colbox}{Satz}[Zusammenhang zwischen reeller und komplexer DFT]
  Sei $\hat{y}=(\hat{y}_0,\dots,\hat{y}_{n-1})^T$ die komplexe DFT von 
  $y=(y_0,\dots,y_{n-1})^T\in\C^n$, wobei $y$ folgende Darstellung hat
  \begin{align*}
    y_j := f(x_{2j}) + i\cdot f(x_{2j+1}), \quad j=0,\dots, n-1
  \end{align*}
  
  Die trigonometrischen Summen $a_k'$ und $b_k'$ (\ref{eq:fftEQ1}) sind gegeben durch
  %
  \begin{align*}
    a_k' - i\cdot b_k'
    &= \tfrac{1}{2}(\hat{y}_k + \overline{\hat{y}_{n-k}}) 
    + \tfrac{1}{2i}(\hat{y}_k - \overline{\hat{y}_{n-k}})e^{-i\pi\cdot\tfrac{k}{n}} \\
    a_{n-k}' - i\cdot b_{n-k}'
    &= \tfrac{1}{2}(\overline{\hat{y}_k} + \hat{y}_{n-k}) 
    + \tfrac{1}{2i}(\overline{\hat{y}_k} - \hat{y}_{n-k})e^{i\pi\cdot\tfrac{k}{n}} \\
  \end{align*}
  %
  für $k=0,\dots n$, wobei $b_0'=b_n'=0$ und $\hat{y}_n=\hat{y}_0$
\end{colbox}

\textit{Beweis.} 
Durch 
%
\begin{align*}
  \overline{\omega_n^{j(n-k)}} 
  = \overline{\underbrace{\omega_n^{jn}}_{1}} \cdot \overline{\omega_n^{-jk}} 
  = \overline{e^{-2\pi i \,/\,n\cdot(-jk)}}
  = e^{-2\pi i\,/\,n\cdot jk}
  = \overline{\omega_n^{jk}} 
\end{align*}
%
erhalten wir für die Summanden der oberen Formel
%
\begin{align*}
  \tfrac{1}{2}(\hat{y}_k + \overline{\hat{y}_{n-k}}) 
  &= \dfrac{1}{2}\cdot\sum_{j=0}^{n-1}
  \Big({y}_j\cdot\omega_n^{jk} + \overline{{y}}_j\cdot\overline{\omega_n^{j(n-k)}}\Big) \\
  &= \dfrac{1}{2}\cdot\sum_{j=0}^{n-1}\left({y}_j+\overline{{y}}_j\right)\cdot\omega_n^{jk}\\
  &= \dfrac{1}{2}\cdot\sum_{j=0}^{n-1}\left({y}_j+\overline{{y}}_j\right)\cdot e^{-2\pi i\cdot\frac{jk}{n}}
\end{align*}
%
und 
%
\begin{align*}
  \tfrac{1}{2i}(\hat{y}_k - \overline{\hat{y}_{n-k}})e^{-i\pi\cdot\tfrac{k}{n}}
  &= \dfrac{1}{2i}\cdot\sum_{j=0}^{n-1}
  \Big({y}_j\cdot\omega_n^{jk} - \overline{{y}}_j\cdot\overline{\omega_n^{j(n-k)}}\Big)
  e^{-i\pi\cdot\tfrac{k}{n}} \\
  &= \dfrac{1}{2i}\cdot\sum_{j=0}^{n-1}
  \left({y}_j-\overline{{y}}_j\right)\cdot\omega_n^{jk}\cdot e^{-i\pi\cdot\tfrac{k}{n}}\\
  &= \dfrac{1}{2i}\cdot\sum_{j=0}^{n-1}
  \left({y}_j-\overline{{y}}_j\right)\cdot e^{-i\pi(2j+1)\tfrac{k}{n}}
\end{align*}
%
Mit Definition von $y_j$ ergibt sich 
%
\begin{align*}
  {y}_j+\overline{{y}}_j &= 2\cdot \mathrm{Re}(y_j) = 2\cdot f(x_{2j}) \\
  {y}_j-\overline{{y}}_j &= 2i\cdot \mathrm{Im}(y_j) = 2\cdot f(x_{2j+1}) \\
\end{align*}
%
und für die Summe
%
\begin{align*}
  &\tfrac{1}{2}(\hat{y}_k + \overline{\hat{y}_{n-k}}) 
  +\tfrac{1}{2i}(\hat{y}_k - \overline{\hat{y}_{n-k}}) e^{-i\pi\cdot\tfrac{k}{n}} \\
  &= \sum_{j=0}^{n-1}\Big(f(x_{2j})e^{-ijk\tfrac{2\pi}{n}}+f(x_{2j+1})e^{-i(2j+1)k\tfrac{\pi}{n}}\Big) \\
  &= \sum_{j=0}^{n-1}\Big(f(x_{2j})\left[\cos(kx_{2j})-i\sin(kx_{2j})\right]
  + f(x_{2j+1})\left[\cos(kx_{2j+1})-i\sin(kx_{2j+1})\right]\Big) \\
  &= \sum_{j=0}^{n-1}\Big(f(x_{2j})\cos(kx_{2j}) + f(x_{2j+1})\cos(kx_{2j+1})\Big)\\
  &\quad - i\cdot\sum_{j=0}^{n-1}\Big(f(x_{2j})\sin(kx_{2j}) + f(x_{2j+1})\sin(kx_{2j+1})\Big) \\
  &= a_k' - ib_k'
\end{align*}
Die zweite Formel des Satzes ergibt sich durch Substitution von $k$ durch $n-k$. 
\qed

Der Vorteil der komplexen DFT ist, dass eine Reduktion gerader Ordnung auf zwei komplexe DFT je der halben
Ordnung möglich ist, führen wir diese Reduktion iterativ durch (was bei einer Potenz von $2$ möglich ist) 
erhalten wir einen Divide \& Conquer Algorithmus mit linear-logarithmischer Laufzeit.

\begin{colbox}{Satz}
  Sei $n=2m$ mit $m\in\N$. Für die komplexen Fourier-Koeffizienten gilt:
  \begin{align*}
    \hat{f}_{2l} &= \sum_{j=0}^{m-1} (f_j+f_{m+j})\omega_n^{2lj} \\
    \hat{f}_{2l+1} &= \sum_{j=0}^{m-1} (f_f-f_{m+j})\omega_n^{j}\cdot\omega_n^{2lj}
  \end{align*}
\end{colbox}

\textit{Beweis.} \\
Für die Einheitswurzeln gilt 
%
\begin{align*}
  \omega_n^{2l(m+j)} 
  = \omega_n^{2lj}\cdot \omega_n^{2lm} 
  = \omega^{2lj}\cdot \big(\omega_n^{2m}\big)^l
  = \omega^{2lj}\cdot \big(\underbrace{\omega_n^n}_{1}\big)^l
  = \omega_n^{2lj} = \omega_m^{lj}
\end{align*}
%
Diese Identität liefert die gewünschte Umformung der Fourier-Koeffizienten:
%
\begin{align*}
  \hat{f}_{2l} 
  &= \sum_{j=0}^{2m-1}f_j\omega_n^{2lj} \\
  &= \sum_{j=0}^{m-1} f_j\omega_n^{2lj} + \sum_{j=0}^{m-1} f_{m+j}\underbrace{\omega_n^{2l(m+j)}}_{\omega_n^{2lj}} \\
  &= \sum_{j=0}^{m-1} (f_j+f_{m+j})\omega_m^{lj}
\end{align*}
und 
\begin{align*}
  \hat{f}_{2l+1} &= \sum_{j=0}^{2m-1} f_j\omega_n^{(2l+1)j} \\
  &= \sum_{j=0}^{m-1}f_j\omega_n^{(2l+1)j} + \sum_{j=0}^{m-1} f_{j+m}\omega_n^{(2l+1)(m+j)}\\
  &= \sum_{j=0}^{m-1}f_j\omega_n^{2lj}\omega_n^j + \sum_{j=0}^{m-1} f_{j+m}
  \underbrace{\omega_n^{2l(m+j)}}_{\omega_n^{2lj}}
  \underbrace{\omega_n^{m}}_{-1}\omega_n^{j}\\
%  &= \sum_{j=0}^{m-1}f_j\omega_n^{(2l+1)j} + \sum_{j=0}^{m-1} f_{j+m}\omega_n^{2l(m+j)}\omega_n^{m}\omega_n^{j}\\
  &= \sum_{j=0}^{m-1}(f_j-f_{m+j})\omega_n^{j}\cdot\omega_m^{lj}
\end{align*}
%
\qed

%\textcolor{green}{Wenn du hier bist, Datei aktualisieren :)}

Dieser Satz ermöglicht es uns nun die Fourier-Transformierte $\hat{f}$ als Kombination von zwei neuen 
Fourier-Transformationen zu schreiben, denn für die Hilfswerte
%
\begin{align*}
  g_{j} := f_j + f_{m+j} 
  \quad \text{und}\quad 
  h_{j} := (f_j) - (f_{m+j})\omega_n^j
\end{align*}
%
gilt 
\begin{align*}
  \hat{g} = (\hat{f}_0,\hat{f}_2,\dots,\hat{f}_{2m-1})^T 
  \quad\text{und}\quad 
  \hat{h} = (\hat{f}_1,\hat{f}_3,\dots,\hat{f}_{2m})^T 
\end{align*}

Wir haben damit die Berechnung einer DFT von $f\in\C^{2m}$ (einmal Ordnung $2m$) auf die Berechnung zweier DFTs von 
$g\in\C^m$ und $h\in\C^m$ (zweimal Ordnung $m$). 

\begin{colbox}{Beispiel}
  Wollen wir eine DFT Ordnung $32$ durchführen, so brechen wir dies erst auf die Berechnung von zwei DFTs mit Ordnung 
  16 herunter. Jede dieser DFTs wird dann wiederum in zwei DFTs mit Ordnung 8 vereinfacht. Wiederholt man dies iterativ 
  so ergibt sich:\\
  $
  \qquad FT_{32} 
  \rightarrow 2\ FT_{16} 
  \rightarrow 4\ FT_{8} 
  \rightarrow 8\ FT_{4} 
  \rightarrow 16\ FT_{2} 
  \rightarrow 32\ FT_{1}
  $
\end{colbox}

\begin{colbox}{Bemerkung}
  Der Rechenaufwand für die Berechnung einer diskreten komplexen Fourier-Transformation nach der Methode des Aufteilens
  entspricht $\mathcal{O}(n\log(n))$.
\end{colbox}
