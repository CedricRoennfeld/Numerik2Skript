% !TeX root = ../script.tex

\section{Die schnelle Fourier-Transformation}
(FFT = Fast Fourier Transformation) \\ \\
\subsection{Fourier-Reihen}
Sei $f$ eine $2\pi$-periodische Funktion, d.h. $f(x+2\pi)=f(x)$ für alle $x\in\mathbb{R}$. Das Ziel ist eine 
Annäherung durch Linearkombinationen $2\pi$-periodischen Funktionen:
\[
    g_n(x) = \dfrac{1}{2}a_0 + \sum_{k=1}^{n}\Big(a_k\cos(kx)+b_k\sin(kx)\Big)
\]
Wir wollen eine Approximation, d.h.
\[
    \|g_n(x)-f(x)\|_2 = \left(\int_{-\pi}^{\pi} (g_n(x)-f(x))^2 dx\right)^{1/2} \to \min
\]
Orthogonalitätsbedingungen der trigonometrischen Funktionen werden verwendet um Fourier-Koeffizienten zu bestimmen:
\begin{align*}
    a_k &= \dfrac{1}{\pi} \int_{-\pi}^{\pi} f(x)\cos(kx) dx, \quad k=0,1,\dots,n \\
    b_k &= \dfrac{1}{\pi} \int_{-\pi}^{\pi} f(x)\sin(kx) dx, \quad k=1,\dots,n 
\end{align*}
Im allg. Fall ergeben sich für die Fourier-Koeffizienten $a_k$ und $b_k$  keine geschlossenen Formeln, d.h. wir 
sind auf numerische Integration angewiesen \\
Die Wahl Trapezregel als Quadratur liefert die diskrete Fouriertransformation \\
Unterteilung des Intervalls $[0,2\pi]$ in $N$ Teilintervalle. Man hat somit eine Schrittweite $h=\tfrac{2\pi}{N}$ und 
Integrationsstützstellen 
\[
    x_j = hj = \dfrac{2\pi}{N}\cdot j, \quad j=0,1,\dots,N
\]
Mit der Trapezregel folgt 
\begin{align*}
    a_k &= \dfrac{1}{\pi} \int_0^{2\pi} f(x) \cos(kx) dx \\
    &\approx\dfrac{1}{\pi} \cdot \dfrac{2\pi}{2N}\left(f(x0)\cdot\cos(kx_0) 
    + 2\sum_{j=1}^{N-1} f(x_j)\cdot\cos(kx_j) + f(x_N)\cdot \cos(kx_N)\right)
\end{align*}

Mit Berücksichtigung der $2\pi$-Periodizität von $f(x)$ ergibt sich für $a_k$ und analog für $b_k$ die Näherungswerte
\begin{align*}
    a_k^* &:= \dfrac{2}{N}\sum{j=1}^N f(x_j)\cdot \cos(kx_j), \quad k=0,1,2,\dots
    b_k^* &:= \dfrac{2}{N}\sum{j=1}^N f(x_j)\cdot \cos(kx_j), \quad k=1,2,3\dots
\end{align*}

\begin{thmbox}{Satz}
    Für die diskreten Stützstellen $x_j$ gelten
    \[
        \sum_{j=1}^N \cos(kx_j) = \begin{cases}
            0, & \text{falls } \tfrac{k}{N}\notin\mathbb{Z} \\
            N, & \text{falls } \tfrac{k}{N}\in\mathbb{Z}
        \end{cases} 
    \]
    und 
    \[
        \sum_{j=1}^N \sin(kx_j) = 0 \text{für alle } k\in\mathbb{Z}
    \]
\end{thmbox}
\textit{Beweis.}
\begin{align*}
    S := \sum_{j=1}^{N} \cos(kx_j) + i\sin(kx_j) = \sum_{j=1}^{N} e^{ikx_j} = \sum_{j=1}^{N} e^{ijkh}
\end{align*}
Dies ist eine endliche geometrische Reihe mit komplexem 
\[
    q := e^{ikh} = e^{2\pi ik/N}
\]
Ist $\tfrac{k}{N}\notin\mathbb{Z}$, dann ist $q\neq 1$ und die Summenformel der geometrischen Reihe ergibt 
\[
     S = e^{ikh}\dfrac{e^{ikhN}-1}{e^{ikh}-1} = e^{ikh}\cdot\dfrac{e^{2\pi ki}-1}{e^{ikh}-1} = 0, \text{wenn } dfrac{k}{N}\notin\mathbb{Z}
\]
Für $\tfrac{k}{N}\in\mathbb{Z}$ folgt wegen $q=1$, dass $S=N$ ist. \\
Betrachtung on Realteilt und Imaginärteil liefert die Behauptung.

\textcolor{red}{Intuition Bilder}

\begin{thmbox}{Satz}
    Die trigonometrischen Funktionen erfüllen für die äquidistanten Stützstellen $x_j$ die diskreten Orthogonalitätsrelationen
    \[
        \sum_{j=1}^{N} \cos(kx_j)\cos(lx_j) = \begin{cases}
            0, &\text{falls } \tfrac{k+l}{N}\notin \mathbb{Z} \text{ und } \tfrac{k-l}{N}\notin \mathbb{Z} \\
            \tfrac{N}{2} &\text{falls entweder } \tfrac{k+l}{N}\in \mathbb{Z} \text{ oder } \tfrac{k-l}{N}\in \mathbb{Z} \\
            N &\text{falls } \tfrac{k+l}{N}\in \mathbb{Z} \text{ und } \tfrac{k-l}{N}\in \mathbb{Z}
        \end{cases}
    \]
    und 
    \[
        \sum_{j=1}^{N} \sin(kx_j)\sin(lx_j) = \begin{cases}
            0, &\text{falls } \tfrac{k+l}{N}\notin \mathbb{Z} \text{ und } \tfrac{k-l}{N}\notin \mathbb{Z} \\
            0 &\text{falls } \tfrac{k+l}{N}\in \mathbb{Z} \text{ und } \tfrac{k-l}{N}\in \mathbb{Z} \\
            -\tfrac{N}{2} &\text{falls } \tfrac{k+l}{N}\in \mathbb{Z} \text{ und } \tfrac{k-l}{N}\notin \mathbb{Z} \\
            \tfrac{N}{2} &\text{falls } \tfrac{k+l}{N}\notin \mathbb{Z} \text{ und } \tfrac{k-l}{N}\in \mathbb{Z}
        \end{cases}
    \]
    und 
    \[
        \sum_{j=1}^{N} \cos(kx_j)\sin(lx_j) = 0 \quad \text{für all } k,l\in\mathbb{N}
    \]
\end{thmbox}
\textit{Beweis.} Zur Überprüfung der Orthogonalitätsrelationen werden die trigonometrischen Identitäten
\begin{align*}
    \cos(kx_j)\cos(lx_j) = \tfrac{1}{2}\Big(\cos(\big(k+l\big)x_j) + \cos(\big(k-l\big)x_j)\Big) \\
    \sin(kx_j)\sin(lx_j) = \tfrac{1}{2}\Big(\cos(\big(k-l\big)x_j) - \cos(\big(k+l\big)x_j)\Big) \\
    \cos(kx_j)\sin(lx_j) = \tfrac{1}{2}\Big(\sin(\big(k+l\big)x_j) - \sin(\big(k-l\big)x_j)\Big) \\
\end{align*}
verwendet und die Aussagen des folgenden Satzes angewendet
\begin{thmbox}{Satz}
    Es sei $N=2n$, mit $n\in\mathbb{N}$. Da Fourier-Polynom
    \[
    g_n^*(x) := \tfrac{1}{2}a^*_0 + \sum_{k=1}^{m}\{a^*_k\cos(kx)+b^*_k\sin(kx)\}
    \]
    von Gram $m<n$ mit Koeffizienten $a_k^*$ und $b^*_k$ approximiert die Funtkion $f(x)$ im diskreten quadratischen 
    Mittel der $N$ Stützstellen $x_j$ derart, dass die Summe der Quadrate der Abweichungen
    \[
    F:=\sum_{j=1}^{N}\{g^*_n(x_j)-f(x_k)\}^2
    \]
    minimal ist.
\end{thmbox}
\begin{egbox}
    Sei $f(x)=x^2$: \\
    \textcolor{red}{$x^2$ Plot} 
\end{egbox}

\subsection{Effiziente Berechnung der Fourier-Koeffizienten}
Die näherungsweise Berechnung der Fourier-Koeffizienten $a_k^*$ und $b_k^*$ ist für eine große Anzahl $N$ der 
Stützstellen sehr aufwendig. \\
Dies ist vorallem bei der diskreten Fouriertransformation relevant, die in Ingenieur- und Naturwissenschaften 
häufig eingesetzt wird, um z.B. die Frequenzen von Vibrationen. \\
$\implies$ Aufwand $\propto N^2$ ($N^2$ trigonometrischen Funktionsauswertungen). Problem bei $N\gg 1000$ \\
Runge 1903, 1905, sowie verchiedene unabhängige Arbeiten vieler Mathematiker.
\subsection{Schnelle Fourier-Transformation (Details)}
Zur Berechnung der Summe
\begin{align*}
    a'_k := \sum_{j=0}^{N-1} f(x_j)\cos(kx_j), \quad k=0,1,\dotsc,\tfrac{N}{2} \\
    b'_k := \sum_{j=0}^{N-1} f(x_j)\cos(kx_j), \quad k=1,2,\dotsc,\tfrac{N}{2}-1 
\end{align*}
mit $x_j=\tfrac{2\pi}{N}\cdot j$, kann für den Spezialfall, in dem $N$ eine Potenz von $2$ ist, ein sehr effizienter 
Algorithmus entwickelt werden, wenn man zu einer komplexen Fouriertransformation übergeht. \\
Aus zwei aufeinanderfolgenden Stützwerten bildet man die $n=N\,/\,2$ komplexen Zahlenwert:
\[y_j := f(x_{2j}) + i\cdot f(x_{2j+1}),\qquad \text{für } j=0,1,\dotsc,n-1\]
\begin{defbox}[Diskrete komplexe Fouriertransformation der Ordnung $n$:]
    \begin{align*}
        c_k := \sum_{j=0}^{n-1} y_j\cdot \exp\left(ijk\dfrac{2\pi}{n}\right) = \sum_{j=0}^{n-1} y_j\cdot \omega_n^{jk}
    \end{align*}
    mit 
    \[\omega_n := \exp\left(-i\cdot\dfrac{2\pi}{n}\right) = \cos\left(\dfrac{2\pi}{n}\right)+i\cdot\sin\left(\dfrac{2\pi}{n}\right)\]
    Dabei sind die $\omega_n$ die $n$-ten Einheitswurzeln.
\end{defbox}
\newpage
\begin{thmbox}{Satz}[Zusammenhang zwischen reellwertigen und komplexen Fourier-Transformation]
    Die reellwertigen trigonometrischen Summen $a_k'$ und $b_k'$ sind gegeben durch die komplexen 
    Fourier-Transformierten $c_k$ durch:
    \begin{align*}
    a_k'-ib_k'&=\tfrac{1}{2}(c_k+\overline{c_{n-k}}) + \tfrac{1}{2i}(c_k-\overline{c_{n-k}})e^{\tfrac{-ik\pi}{n}} \\
    a_{n-k}'-ib_{n-k}'&=\tfrac{1}{2}(\overline{c_k}+c_{n-k}) + \tfrac{1}{2i}(\overline{c_k}-c_{n-k})e^{\tfrac{ik\pi}{n}}
    \end{align*}
    für $k=0,\dots n$ falls $b_0'=b_n'=0$ und $c_n=c_0$
\end{thmbox}
\textit{Beweis.} 
Für den ersten Summanden der oberen Formel erhält man 
\begin{align*}
    \dfrac{1}{2}\left(c_k+\overline{c_{n-k}}\right) 
    &= \dfrac{1}{2}\cdot\sum_{j=0}^{n-1}\left\{y_j\cdot\omega_n^{jk} + \overline{y_j}\cdot\overline{\omega_n^{j(n-k)}}\right\} \\
    &= \dfrac{1}{2}\cdot\sum_{j=0}^{n-1}\left(y_j+\overline{y_j}\right)\cdot\omega_n^{jk}
\end{align*}
Für den Ausdruck in Klammern des 2. Summanden \textcolor{red}{prüfen, glaube - statt +}
\begin{align*}
    \dfrac{1}{2i}\left(c_k+\overline{c_{n-k}}\right) 
    &= \dfrac{1}{2i}\cdot\sum_{j=0}^{n-1}\left\{y_j\cdot\omega_n^{jk} + \overline{y_j}\cdot\overline{\omega_n^{j(n-k)}}\right\} \\
    &= \dfrac{1}{2}\cdot\sum_{j=0}^{n-1}\left(y_j+\overline{y_j}\right)\cdot\omega_n^{jk}
\end{align*}
Mit Definition von $y_j$ folgt
\begin{align*}
    &\dfrac{1}{2}\left(c_k+\overline{c_{n-k}}\right) - \dfrac{1}{2i}\left(c_k+\overline{c_{n-k}}\right)e^{-\tfrac{ik\pi}{n}}\\
    &= \sum_{j=0}^{n-1}\left\{f(x_{2j})e^{-ijk\tfrac{2\pi}{n}}+f(x_{2j+1})e^{-i(2j+1)k\tfrac{\pi}{n}}\right\} \\
    &= \sum_{j=0}^{n-1}\left\{f(x_{2j})\left[\cos(kx_{2j})-i\sin(kx_{2j})\right]
    + f(x_{2j+1})\left[\cos(kx_{2j+1})-i\sin(kx_{2j+1})\right]\right\} \\
    &= a_k' - ib_k'
\end{align*}
Die zweite Formel des Satzes ergibt sich durch Substitution von $k$ durch $n-k$. 
\qed\\ \\
Die Reduktion einer komplexen Fouriertransformation von gerader Ordnung auf zwei Fouriertransformationen je der halben
Ordnung ist möglich. \\
Diese Reduktion der Ordnung wird iterativ durchgeführt: \\ \\
Es sei $n=2m$, $m\in\mathbb{N}$. Dann gilt für die komplexe Fouriertransformierte $c_k$ mit geraden Indizes 
$k=2l$, $l=0,1,\dots,m-1$:
\begin{align*}
    c_{2l} = \sum_{j=0}^{2m-1}y_j\omega_n^{2lj} = \sum_{j=0}^{m-1} (y_j+y_{m+j})\omega_n^{2lj}
\end{align*}
Dabei wurde die Identität 
\[
    \omega_n^{2l(m+j)} 
    = \omega_n^{2lj}\cdot \omega_n^{2lm} 
    = \omega^{2lj}\cdot \omega_n^{ln} = \omega_n^{2lj}\cdot \left(e^{-i\tfrac{2\pi}{n}}\right)^{2ln} 
    = \omega_n^{2lj}\cdot \left(e^{-i\cdot2\pi}\right)^l 
    = \omega_n^{2lj}
\]
Mit den $m$ Hilfswerten 
\[
    z_j := y_j + y_{m+j}, \quad j=0,\dots,m-1
\]
und wegen $\omega_n^2 = \omega_m$ sind die Koeffizienten 
\[
    c_{2l} = \sum_{j=0}^{m-1} z_jw_m^{jl}
\]
die Fouriertransformierten der Ordnung $m$ der Hilfswerte $z_j$. \\
Für die $c_k$ mit ungeraden Indizes $k=2l+1$ mit $l=0,1,\dots,m-1$ gilt 
\begin{align*}
    c_{2l+1} &= \sum_{j=0}^{2m-1} y_j\omega_n^{(2l+1)j} \\
    &= \sum_{j=0}^{m-1}\left\{ y_j\omega_n^{(2l+1)j}+y_{j+m}\omega_n^{(2l+1)(n+j)}\right\}\\
    &= \sum_{j=0}^{m-1}(y_j-y_{m+j})\omega_n^{(2l+1)j} \\
    &= \sum_{j=0}^{m-1}(y_j-y_{m+j})\omega_n^{j}\cdot\omega_n^{2lj}
\end{align*}
Mit den weiteren $m$ Hilfswerten
\[
    z_{m+j} := (y_j-y_{m+j})\omega_n^j,\quad j=0,1,\dots,m-1
\]
sind die Koeffizienten
\[
    c_{2l+1} = \sum_{j=0}^{m-1} z_{m+j} \omega_m^{jl}, \quad l=0,1,\dots,m-1
\]
die Fouriertransformierten der Ordnung $m$ der Hilfswerte $z_{m+j}$. \\
Die Zurückführung einer komplexen Fouriertransformationen der Ordnung $n=2m$ auf $2$ komplexe Fouriertransformationen 
der Ordnung $m$ erfordert nach den obigen Formeln als wesentlichen Rechenaufwand $m$ komplexe Multiplikationen. \\
In die Ordnung $n=2^\gamma$, $\gamma\in\mathbb{N}$, so kann die Reduktion auf $2$ Fouriertransformationen halber Ordnung
iterativ durchgeführt werden. \\
\begin{egbox}
    $FT_{32} \to 2FT_{16}\to 4FT_8 \to 8FT_4 \to 16FT_2\to 32FT_1$
\end{egbox}
Da jeder Schritt $n\,/\,2$ komplexe Multiplikationen fordert, beträgt der gesamte Rechenaufwand
\[
    z_{FT,n} = \tfrac{1}{2}n\gamma = \tfrac{1}{2}n\log_2 n
\]
