% !TeX root = ../script.tex

\section{Krylov-Raum-Methoden für EW-Probleme}
Wir verfolgen die gleiche Idee, wie auch schon bei linearen Gleichungssystemen, d.h. die ursprünglich hochdimensionalen
Probleme, werden durch geeignete Unterräume (Krylov-Räume) in kleinere Probleme umgewandelt. \\
Wir erhalten dabei ein iteratives Vorgehen, zu betrachtende Beispiele sind die Arnoldi-Methode und die 
Lanczos-Methode.

Wir betrachten also die Eigenwertgleichung $Az=\lambda z$ mit $A\in\C^{n\times n}$ 
(ab jetzt erlauben wir auch komplexe Matrizen), wobei $A$ eine sehr große
Matrix ist, typischerweise $n\geq 10^4$.

\subsection{Galerkin-Approximation}
Eigenwertprobleme können äquivalent in Variationsform (schwache Formulierung) geschrieben werden, diese besagt: \\
$z\in\C^n$ ist genau dann ein Eigenvektor von $A$ zum Eigenwert $\lambda\in\C$, wenn
%
\begin{align*}
  \langle Az,y\rangle_2 
  = \lambda\langle z,y\rangle_2 \quad\forall\,y\in\C^n 
  \tag{1}\label{eq:galerkinEQ1}
\end{align*}
%
Diese Äquivalenz gilt, da aus $\langle r, y\rangle_2 = 0$ für alle $y\in\C^{n}$ folgt, dass $r=0$ sein muss, 
in unserem Fall ist $r=Az-\lambda z$ das Residuum des Eigenwertproblems.

Sei $K_m=\Span\{q^{(1)},\dots,q^{(m)}\}$ ein geeignet gewählter Unterraum von $\C^n$ kleiner Dimension, d.h. 
$\dim K_m=m\ll n$, dann wird das $n$-dimensionale Eigenwertproblem (\ref{eq:galerkinEQ1}) mit 
folgendem $m$-dimensionalem Eigenwertproblem approximiert: 
%
\begin{align*}
  \text{suche } z\in K_m, \ \lambda\in\C : 
  \quad \text{mit}\quad 
  \langle Az,y\rangle_2 
  = \lambda\langle z,y\rangle_2 
  \quad\forall\,y\in K_m
\end{align*}
%
Aufgrund der Bilinearität des Skalarproduktes reicht es auch, wenn wir nur die erzeugenden $q^{(i)}$ betrachten, 
statt alle $y\in K_m$. 

Wir entwickeln die Eigenvektoren $z\in K_m$ bzgl. der gegebenen Basis:
%
\begin{align*}
  z = \sum_{j=1}^{m} \alpha_j q^{(j)}
\end{align*}
%
und erhalten somit das Galerkin-System
%
\begin{align*}
  \sum_{j=1}^{k} \alpha_j \langle Aq^{(j)},q^{(i)}\rangle_2 
  = \lambda\cdot\sum_{j=1}^{k} \alpha_j \langle q^{(j)},q^{(i)}\rangle_2
  \qquad\forall\,i=1,\dots,m
\end{align*}
Dabei charakterisieren die $\alpha_i$ unser gesuchtes $z$, wir schreiben dieses System daher typischerweise 
in kompakter Form als Eigenwertproblem
%
\begin{align*}
  \mathcal{A}\alpha = \lambda\mathcal{M}\alpha
\end{align*} 
%
mit Vektoren $\alpha=(\alpha_1,\dots,\alpha_m)$ und Matrizen
$\mathcal{A}=(\langle Aq^{(j)}, q^{(i)}\rangle_2)_{i,j=1}^m$, 
$\mathcal{M} = (\langle q^{(j)}, q^{(i)}\rangle_2)_{i,j=1}^m$.

Im Folgenden betrachten wir immer die \textit{kartesische Repräsentation} der Basisvektoren $q^{(i)}=(q_j^{(i)})_{j=1}^n$ 
und somit schreibt man das Galerkin-EW-Problem in der Form\footnote{
  Als Erinnerung: Im Komplexen ist das Standardskalarprodukt definiert durch 
  $\langle x,y\rangle_2 = \sum_i \overline{x}_i\cdot y_i$
}
%
\begin{align*}
  \sum_{j=1}^{m} \alpha_j\cdot \sum_{k,l=1}^{n} a_{k,l}\cdot \overline{q_k}^{(j)}\cdot q^{(i)}_l 
  = \lambda\cdot \sum_{j=1}^{m} \alpha_j \cdot\sum_{k,l=1}^{n}\overline{q_k}^{(j)}\cdot q^{(i)}_l
  \quad\forall\, i=1,\dots,m
\end{align*}
%
Mit $\mathcal{Q}^{(m)}=[q^{(1)},\dots,q^{(m)}]\in\C^{n\times m}$ kann dies in kompakter Form
%
\begin{align*}
  \mathcal{Q}^{(m)H}A\mathcal{Q}^{(m)}\alpha = \lambda \mathcal{Q}^{(m)H}\mathcal{Q}^{(m)}\alpha
\end{align*}
formuliert werden.

Wenn $\{q^{(1)},\dots,q^{(m)}\}$ eine Orthonormalbasis von $K_m$ ist, reduziert sich dies zum normalen EW-Problem:
%
\begin{align*}
  \underbrace{\mathcal{Q}^{(m)H}A\mathcal{Q}^{(m)}}_{=: H^{(m)}\in\C^{m\times m}}\alpha 
  = \lambda \alpha 
  \tag{2}\label{eq:galerkinEQ2}
\end{align*}
%
Falls $H^{(m)}$ eine spezielle Struktur hat (z.\,B. Hessenberg-Matrix oder symmetrische Tridiagonalgestalt), dann kann 
das EW-Problem mit niedriger Dimension (\ref{eq:galerkinEQ2}) mit z.B. QR-Methode gelöst werden. 

Seine Eigenwerte, genannt \textit{Ritz-Eigenwerte}, können als Approximationen der dominanten Eigenwerte der 
ursprünglichen Matrix $A$ betrachtet werden. 

\begin{colbox}{Bemerkung}[Krylov-Methode] 
  \begin{enumerate}
    \item[1.] Wähle geeignete Unterräume $K_m\in\C  ^{m\times m}$, $m\ll n$ (Krylov-Räume) durch Verwendung der 
    Matrix $A$ und deren Potenz.
    \item[2.] Konstruiere eine Orthonormalbasis $\{q^{(1)},\dots, q^{(m)}\}$ von $K_m$ mit der stabilisierten 
    Version des Gram-Schmidt-Algorithmus und setze $\mathcal{Q}^{(m)}:=[q^{(1)},\dots,q^{(m)}]$.
    \item[3.] Berechne die Matrix $H^{(m)}:=\mathcal{Q}^{(m)H}A\mathcal{Q}^{(m)}$, welche 
    konstruktionsbedingt eine Hessenberg-Matrix oder im hermiteschen Fall eine hermitesche Tridiagonalmatrix ist. 
    \item[4.] Löse das Eigenwertproblem der reduzierten Matrix $H^{(m)}\in\C^{m\times m}$ durch die 
    QR-Methode.
    \item[5.] Die Eigenwerte von $H^{(m)}$ als Näherung der dominanten (betragsgrößten) Eigenwerte 
    von $A$. Im Falle des betragskleinsten Eigenwert, muss die Matrix $A^{-1}$ betrachtet werden (Konstruktion 
    der Unterräume $K_m$ kann sehr aufwendig sein).
  \end{enumerate}
\end{colbox}

\subsection{Arnoldi-Methode}
\textbf{Idee:} \\
Die Potenzmethode verwendet nur die aktuelle Iterierte $A^mq$ mit $m\ll n$ für den normierten Startvektor 
$q\in\C^n$ mit $\vertn{2}{q}_2=1$, ignoriert aber die bereits berechneten Iterierten $\{q,Aq,A^2q,\dots,A^{m-1}q\}$.

Wir wollen diese bereits bestimmten Informationen nun nutzen und erstellen eine sogenannte \textit{Krylov-Matrix}:
%
\begin{align*}
  K_m = [q,Aq,A^2q,\dots,A^{m-1}q]
  \quad\text{mit }1\leq m\leq n
\end{align*}
%
Die Spalten dieser Matrix sind jedoch nicht orthogonal zueinander, außerdem konvergiert $A^tq$ gegen den 
Eigenvektor zum betragsgrößten Eigenwert, d.h. $K_m$ ist schlecht konditioniert\footnote{
  Für nicht-invertierbare Matrizen $A\in\C^{n\times m}$ ist die Konditionszahl über das Pseudoinverse definiert
}, da sich die letzten Spalten kaum ändern. 

Wie wir sehen werden, ist die Konstruktion in eine orthogonale Basis mit dem Gram-Schmidt-Algorithmus instabil, 
wir wählen daher als Alternative in der Arnoldi-Methode die Verwendung einer stabilisierten Variante des 
Gram-Schmidt-Verfahrens um eine Folge orthonormaler Vektoren $\{q^{(1)},q^{(2)},\dots\}$ (bezeichnet als 
Arnoldi-Vektoren) zu erzeugen, sodass für jedes $m$ die Vektoren $\{q^{(1)},\dots,q^{(m)}\}$ den Krylov-Unterraum $K_m$ 
aufspannen. 

\begin{colbox}{Definition}
Für das Folgende definieren wir den orthogonalen Projektionsoperator:
%
\begin{align*}
  \text{proj}_u(v) 
  := \dfrac{\langle v,u\rangle_2}{\vertn{2}{u}_2^2}\cdot u
\end{align*}
%
Dieser projiziert den Vektor $v$ auf $\Span\{u\}$.
\end{colbox}

Mit diesem Operator ergibt sich das \textit{klassische Gram-Schmidt-Orthogonalisierungs-Verfahren} als 
%
\begin{align*}
  q^{(1)} 
  &= \dfrac{q}{\vertn{2}{q}_2}, \\
  \text{und für } &t=2,\dots,m: \\
  \tilde{q}^{(t)} 
  &= A^{t-1}q - \sum_{j=1}^{t-1} \text{proj}_{q^{(j)}}(A^{t-1}q), 
  \\
  q^{(t)} 
  &= \dfrac{\tilde{q}^{(t)}}{\vertn{2}{\tilde{q}^{(t)}}_2}
\end{align*}
%
Der $t$-te Schritt projiziert dabei die Komponente von $A^{t-1}q$ in Richtung der bereits bestimmten orthogonalen 
Vektoren $\{q^{(1)},\dots,q^{(t-1)}\}$. 

Dieses Vorgehen ist jedoch numerisch instabil, denn Rundungsfehler in frühen Orthogonalisierungsschritten 
pflanzen sich bei der sukzessiven Subtraktion fort und verstärken sich von Vektor zu Vektor, 
sodass die resultierende Basis numerisch nicht mehr orthogonal ist. 

Wir betrachten daher das \textit{modifizierte Gram-Schmidt-Verfahren}, wobei der $t$-te Schritt die Komponente
von $Aq^{(t)}$ in Richtung $\{q^{(1)},\dots,q^{(t-1)}\}$ projiziert:
%
\begin{align*}
  q^{(1)} 
  &= \dfrac{q}{\vertn{2}{q}_2},\\
  \text{und für } &t=2,\dots,m: \\
  \tilde{q}^{(t)} 
  &= Aq^{(t-1)} - \sum_{j=1}^{t-1} \text{proj}_{q^{(j)}}(Aq^{(t-1)}), 
  \tag{1}\label{eq:arnoldiEQ1}\\
  q^{(t)} &
  = \dfrac{\tilde{q}^{(t)}}{\vertn{2}{\tilde{q}^{(t)}}_2}
\end{align*}
%
Nach Konstruktion ist $q^{(t)}$ senkrecht zu $\{q^{(j)}\}_{j=1}^{t-1}$ und damit auch zu 
$\{\text{proj}_{q^{(j)}}(Aq^{(t-1)})\}_{j=1}^{t-1}$, es folgt also
% 
\begin{align*}
  \langle q^{(t)}, \tilde{q}^{(t)}\rangle_2 
  %= \vertn{2}{\tilde{q}^{(t)}}_2 
  = \left\langle q^{(t)},Aq^{(t-1)} - \sum_{j=1}^{t-1} \text{proj}_{q^{(j)}}(Aq^{(t-1)})\right\rangle _2 
  = \langle q^{(t)}, Aq^{(t-1)}\rangle_2
\end{align*}
%
Durch die Setzung $h_{i,t-1} := \langle Aq^{(t-1)},q^{(i)} \rangle_2$ ergibt sich mit 
dem modifizierte Gram-Schmidt-Algorithmus dann
%
\begin{align*}
  Aq^{(t-1)}
  =\sum_{i=1}^{t} h_{i,t-1 }q^{(i)}, 
  \qquad t=2,\dots,m+1
\end{align*}
%
In der Praxis wird der modifizierte Gram-Schmidt-Algorithmus in der folgenden iterierten Form implementiert:
%
\begin{align*}
  q^1 &= \vertn{2}{q}_2^{-1}q, \\
  q^{(t,1)} &= Aq^{(t-1)}, \\
  q^{(t,j+1)} &= q^{(t,j)}-\text{proj}_{q^{(j)}}(q^{(t,j)}), 
  \tag{2}\label{eq:arnoldiEQ2}\\
  q^{(t)} &= \vertn{2}{q^{(t,t)}}_2^{-1}q^{(t,t)}
\end{align*}
%
Wir erhalten dabei das gleiche Resultat, wie beim klassischen Gram-Schmidt-Verfahren, 
aber mit kleinerem numerischen Fehler, da sich die Fehler nicht fortpflanzen.

\begin{colbox}{Definition}[Arnoldi-Algorithmus] \ \\
  Für eine beliebige Matrix $A\in\C  ^{n\times n}$ bestimmt die Arnoldi-Methode eine Folge orthonormaler 
  Vektoren $q^{(t)}\in\C  $ für $1\leq t \leq m \ll n$ (Arnoldi-Basis), durch Anwendung der modifizierten 
  Gram-Schmidt-Methode (2) auf die Basis $\{q,Aq,A^{m-1}q\}$ des Krylov-Unterraums $K_m$.
\end{colbox}

\textcolor{red}{(Algobox)} \\
Startvektor: $q^{(1)}=\vertn{2}{q}^{-1}_2 q$ \\
Iteriere für $2\leq t\leq m: q^{t,1}=Aq^{(t-1)}$ \\
und für $j=1,\dots,t-1: h_{j,t} = \langle q^{t,j},q^{(j)}\rangle_2$ und $q^{t,j+1=q^{t,j}-h_{j,t}q^{(j)}}$ und $h_{t,t}=\vertn{2}{q^{t,t}}_2$ 
und $q^{(t)} = h_{t,t}^{-1}\cdot q^{t,t}$ \\ \\
Bezeichne die $n\times m$-Matrix aus den ersten Arnoldi-Vektoren $\{q^1,q^2,\dots,q^m\}$ mit 
\begin{align*}\mathcal{Q}^{(m)}:=[q^1,q^2,\dots,q^m]\end{align*} und sei $H^{(m)}$ die obere Hessenberg Matrix ($m\times m)$ aus $h_{j,k}$:
\begin{align*}H^{(m)} = \begin{pmatrix}
  h_{11} & h_{12} & h_{13} & \dots & h_{1m} \\
  h_{21} & h_{22} & h_{23} & \dots & \vdots \\
  0 & h_{32} & h_{33} & \dots & \vdots \\
  \vdots & \ddots &  \ddots &   \ddots &  h_{m-1,m} \\
  0 &\dots & 0 & h_{m,m-1} & h_{m,m}
\end{pmatrix}\end{align*}
Die Matrizen $\mathcal{Q}^{(m)}$ sind orthonormal und mit (1) ergibt sich die Arnoldi-Beziehung 
\begin{align*}A\mathcal{Q}^{(m)} = \mathcal{Q}^{(m)} H^{(m)} + h_{m,m+1}[0,\dots,0,q^{m+1}]\tag{3}\end{align*}
Multiplikation mit ${\mathcal{Q}^{(m)}}^*$  und Verwendung von 
\begin{align*}{\mathcal{Q}^{(m)}}^* \mathcal{Q}^{(m)} = I \quad \text{und}\quad {\mathcal{Q}^{(m)}}^* q^{m+1}=0\end{align*}
ergibt 
\begin{align*}H^{(m)} = {\mathcal{Q}^{(m)}}^* A \mathcal{Q}^{(m)}\end{align*}
Im Grenzfall $m=n$ ist die Matrix $H^{(n)}$ ähnlich zu $A$ und hat die gleichen Eigenwerte. \\
Dies legt nahe, dass auch für $m\ll n$ die Eigenwerte der reduzierten Matrix $H^{(m)}$ eine gute Approximation 
einiger Eigenwerte von $A$ sind. Wenn der Algorithmus endet (in exakter Arithmetik) für $m<n$ mit $h_{m,m+1}$ dann
ist der Krylov-Raum $K_m$ ein invarianter Unterraum der Matrix $A$ und die reduzierte Matrix $H^{(m)} = 
{\mathcal{Q}^{(m)}}^* A \mathcal{Q}^{(m)}$ hat $m$ Eigenwerte gemeinsam mit $A$, d.h. $\sigma(H^{(m)})\subset \sigma(A)$\footnote{Beweis: Übungsblatt}
Das folgende Lemma liefert a posteriori Abschätzungen der Genauigkeit für die Approximation der Eigenwerte von $A$ durch 
$H^{(m)}$.
\begin{thmbox}{Lemma}
  Sei $\{\mu,w\}$ ein Eigenpaar der Hessenberg-Matrix $H^{(m)}$ und sei $v=\mathcal{Q}^{(m)}w$ sodass $\{\mu,v\}$ ein 
  approximiertes Eigenpaar von $A$ ist. Dann gilt
  \begin{align*}\vertn{2}{Av-\mu w}_2 = |h_{m+1,m}|\cdot |w_m|,\end{align*} 
  wobei $w_m$ die letzte Komponente des Eigenvektors $w$ ist.
\end{thmbox}
\textit{Beweis.} Multiplikation von (3) mit $w$ ergibt 
\begin{align*}
Av &= A\mathcal{Q}^{(m)}w\\ 
&= \mathcal{Q}^{(m)}H^{(m)}w + h_{m+1,m}\cdot[0,\dots,0,q^{m+1}]w \\
&= \mu \mathcal{Q}^{(m)}w + h_{m+1,m}\cdot[0,\dots,0,q^{m+1}]w \\
&= \mu v + h_{m+1,m}\cdot[0,\dots,0,q^{m+1}]w
\end{align*}
Daraus folgt mit $\vertn{2}{q_{m+1}}_2 = 1$, dass
\begin{align*}\vertn{2}{Av-\mu v}_2 = |h_{m+1,m}|\cdot |w_m|\end{align*}
\qed \\ \\
Dies liefert keine a priori-Information der Konvergenz der Eigenwerte von $H^{(m)}$ gegen die von $A$ für $m\to n$, aber
liefert a posteriori-Prüfung basierend auf den berechneten Größen $h_{m+q,m}$ und $w_m$, ob das erhaltene Paar 
$\{\mu,w\}$ eine gute Approximation ist.
\begin{rembox}
  Die Ritz-Eigenwerte konvergieren zu den betragsgrößten Eigenwerten von $A$. Falls die betragskleinsten Eigenwerte 
  bestimmt werden sollen, muss das diskutierte Verfahren auf die inverse Matrix angewendet werden (Vgl. Inverse 
  Iteration nach Wielandt). In diesem Fall hat man einen großen Aufwand die Krylov-Räume 
  $K_m = \Span{q,A^{-1}q,\dots,A^{-m+1}q}$ zu bestimmen, da hierfür die linearen Systeme $v^0:=q, Av^1=v^0, 
  \dots, Av^m=v^{m-1}$ sukzessiv gelöst werden müssen.
\end{rembox}
\begin{rembox}
  Typische Implementierungen der Arnoldi-Methode werden nach einer bestimmten Anzahl von Iterationen neu begonnen.
  Es kann untersucht werden, dass die Konvergenz sich mit einer größeren Krylov-Unterraum-Dimension $m$ verbessert.
  Die Größe $m$, für die eine optimale Konvergenz erhalten wird, ist leider nicht im Voraus bekannt. \\
  $\implies$\glqq{}Switching\grqq{}-Strategien um zu testen, ob ein Neustart sinnvoll ist, 
  um die Konvergenz zu beschleunigen.
\end{rembox}
\begin{rembox}
  Der Algorithmus der rekursiven Form des Gram-Schmidt-Verfahrens kann auch für die stabile Orthogonalisierung einer 
  allgemeinen Basis $\{v_1,\dots,v_m\}\subset\C  ^n$ verwendet werden:\\
  \textcolor{red}{Algobox} \\
  $u^1=\vertn{2}{v^1}_2^{-1}\cdot v^1$ \\
  Für $t=2,...,m$:  \\
  Für $j=1,...,t-1$: \\
  $u^{t,1}=v^t, u^{t,j+1}=u^{t,j}-proj_{u^{j}}(u^{t,j})$ \\
  $u^t=\vertn{2}{u^{t,t}}_2^{-1}\cdot u^{t,t}$ \\
  Dieser modifizierte Gram-Schmidt-Algorithmus (mit exakter Arithmetik) liefert das gleiche Resultat, 
  wie der klassiche Gram-Schmidt-Algorithmus. \\
  $u^1=\vertn{2}{v^1}_2^{-1}\cdot v^1$ \\
  Für $t=2,...,m$: \\
  $\tilde{u}^t = v^t - \sum_{j=1}^{t-1} proj_{u^j}(v^t)$ \\
  $u^t = \vertn{2}{\tilde{u}}_2^{-1}\cdot\tilde{u}^t$
  Beide Algorithmen haben die gleiche arithmetische Komplexität. \\
  In jedem Schritt wird ein zu den vorherigen Vektoren orthogonaler Vektor bestimmt und dieser ist auch orthonormal
  zu einem eventuellen Fehler, der bei den Berechnungen entstand, was die Stabilität angeht.\\
  Für die Fehlerabschätzung der resultierenden orthonormalen Matrix $u=[u^1,\dots,u^m]$ gilt:
  \begin{align*}
    \vertn{2}{u^Tu-I}_2 \leq \dfrac{c_1\cdot \Cond_2(A)}{1-c_2\cdot\Cond_2(A)}
  \end{align*}
  (Vgl. Björk \& Paige)
\end{rembox}
\begin{rembox}
  Andere Methoden zu Orthogonalisierung (wie z.B. Householder Transformation oder Givens-Rotation) sind zum Teil 
  stabiler, als die stabilisierte Gram-Schmidt-Methode, aber aufgrund der iterativen Anwendungsmöglichkeit ist 
  Gram-Schmidt beim Arnoldi-Verfahren vorteilhafter.
\end{rembox}

\subsection{Lanczos-Methode}
Voraussetzung: $A$ sei hermitesch. Dann erhält man für die Rekursions-Formel der Arnoldi-Methode:
\begin{align*}
  \tilde{q}^{(t)} = Aq^{(t-1)} - \sum_{j=1}^{t-1} \langle Aq^{(t-1)}, q^{(j)}\rangle_2 q^{(j)}, \quad \text{für } t=2,\dots,m+1
\end{align*}
da $\langle Aq^{(t-1)}, q^{(j)}\rangle_2 = \langle q^{(t-1)}, Aq^{(j)}\rangle_2 = 0$ für $j=1,\dots,t-3$. \\
Die Vereinfachung zu 
\begin{align*}\tilde{q}^{(t)} = Aq^{(t-1)} - \underbrace{\langle Aq^{(t-1)}, q^{(t-1)}\rangle_2}_{=:\alpha_{t-1}} q^{(t-1)}
- \underbrace{\langle Aq^{(t-1)}, q^{t-2}\rangle_2}_{=:\beta_{t-2}} q^{t-2} = Aq^{(t-1)}-\alpha_{t-1}q^{(t-1)}-\beta_{t-2}q^{t-2}\end{align*}
Da $A$ hermitesch, ist $\alpha_{t-1}\in\R$. Multiplikation mit $q^{(t)}$ ergibt
\begin{align*}
  \vertn{2}{\tilde{q}^{(t)}}_2 = \langle q^{(t)}, \tilde{q}^{(t)}\rangle_2 = \langle q^{(t)}, Aq^{(t-1)}-\alpha_{t-1}q^{(t-1)}-\beta_{t-2}q^{t-2}\rangle_2 
  = \langle q^{(t)}, Aq^{(t-1)}\rangle_2 = \langle Aq^{(t)}, q^{(t-1)}\rangle_2 = \beta_{t-1}
\end{align*}
Daraus folgt, dass auch $\beta_{t-1}\in\R$ und $\beta_{t-1}q^{(t)} = \tilde{q}^{(t)}$. Also erhält man 
\begin{align*}
  Aq^{(t-1)} = \beta_{t-1}q^{(t)} + \alpha_{t-1}q^{(t-1)} + \beta_{t-2}q^{t-2},\quad\text{für } t=2,\dots,m+1
\end{align*}
In Matrix-Form:

\begin{align*}A\cdot \mathcal{Q}^{(m)} = \mathcal{Q}^{(m)}\cdot\underbrace{\begin{pmatrix}
  \alpha_1 & \beta_2 & & & & \\
  \beta_2 & \alpha_2 & \beta_3 & & & \\
  & \beta_3 & \alpha_3 & \ddots & & \\
  & & & \ddots & \beta_{m-1} & \\
  & & & \beta_{m-1} & \alpha_{m-1} & \beta_m \\
  & & & & \beta_m & \alpha_m
\end{pmatrix}}_{=:T^{(m)}} + \beta_m\cdot[0,\dots,0,q^{m+1}]\end{align*}
wobei die Matrix $T^{(m)}\in\R^{m\times m}$ ist reell und symmetrisch. Von dieser sogenannten Lanczos-Beziehung
erhält man 
\begin{align*}(\mathcal{Q}^{(m)})^* A \mathcal{Q}^{(m)} = T^{(m)}\end{align*} 
\begin{defbox}[Lanczos-Algorithmus]
  Für eine hermitische Matrix $A\in\C  ^{n\times n}$ bestimmt die Lanczos-Methode eine Menge von orthogonalen 
  Vektoren $\{q^1,\dots,q^m\}$, $m\ll n$ durch Anwendung der Gram-Schmidt-Methode auf die Basis 
  $\{q,Aq,\dots,A^{m-1}q\}$ des Krylov-Raumes $K_m$:
\end{defbox}
\textcolor{red}{Algobox:} \\
Startwerte: $q^1=\vertn{2}{q}_2^{-1}q, q^0=0, \beta_1=0$ \\
Iteriere für $1\leq t\leq m-1$:
\begin{align*}
  r^t = Aq^{(t)}, \quad \alpha_t = \langle r^t, q^{(t)}\rangle_2 \\
  s^t = r^t - \alpha_t q^{(t)} - \beta-tq^{(t-1)} \\
  \beta^{t+1} = \vertn{2}{s^t}_2, q^{t+1} = \beta_{t+1}^{-1}s^t \\
  r^m = Aq^m, \alpha_m = \langle r^m,q^m\rangle_2
\end{align*}
Nachdem die Matrix $T^{(m)}$ berechnet ist, kann ihr Eigenwert $\lambda_i$ und der zugehörige Eigenvektor $w^{i}$
bestimmt werden, z.B. mit QR-Algorithmus. \\
Die Eigenwert und Eigenvektoren von $T^{(m)}$ werden mit dem Aufwand $\mathcal{O}(m^2)$ berechnet. Die Eigenwerte 
approximieren die der ursprünglichen Matrix. \\
Die Ritz Eigenvektoren $v^i$ von $A$ können mit $v^i=\mathcal{Q}^{(m)}\cdot w^i$ berechnet werden.

\subsection{Pseudospektren}
Motivation:
Mit Fußball spielen: \textcolor{red}{Bild einfügen} \\
Bei dem einen stabil, weil Ball bleibt in Kuhle. Bei dem anderen instabil, da Ball wegrollt.
\\
Sauber: Fehler verschwinden / werden schlimmer. \\ \\
Spezialfall: \textcolor{red}{Bild einfügen} \\
Kleine Auslenkung ist kein Problem, wenn zu groß, dann schon. \\
Frage ist, wie groß ist das Attraktorgebiet? \\ \\ \\

Begriff der Pseudospektren geht auf Landau zurück. Viele Resultate von Trefethen etal. \\
Konzept des Pseudo-Spektrums ist bei normalen Operatoren die Vereinigung der $\varepsilon$-Umgebungen seiner Eigenwerte

\begin{defbox}
  Für $\varepsilon\in\R_+$, ist das $\varepsilon$-Pseudospektrum $\sigma_\varepsilon(A)\subset\C  $ 
  einer Matrix $A\in\K^{n\times n}$ definiert als 
  \begin{align*}
    \sigma_\varepsilon(A) := \{z\in\C  \sigma(A)\,|\, \vertn{2}{(A-zI)^{-1}}_2\geq \varepsilon^{-1}\} \cup \sigma(A)
  \end{align*}
\end{defbox}

\begin{rembox}
  Das Konzept des Pseudo-Spektrums kann in viel allgemeineren Situationen eingeführt werden. 
  (z.B. Dunford \& Schwartz oder Kato)
\end{rembox}

\begin{rembox}
  Krylov-Unterraum-Methoden, die bisher diskutiert wurden, lassen sich zur Berechnung des Pseudo-Spektrums einer 
  Matrix verwenden. (z.B. bei Matrix von diskretisierten partiellen Differentialgleichungen)
\end{rembox}

\begin{thmbox}{Lemma}
  \begin{enumerate}
    \item Das $\varepsilon$-Pseudospektrum einer Matrix $T\in\C  ^{n\times n}$ kann definiert werden 
      durch
      \begin{align*}\sigma_\varepsilon(T) := \{z\in\C  \,|\,\sigma_{\min}(zI-T)\leq\varepsilon\}\end{align*}
      wobei $\sigma_{\min}(B)$ den kleinsten Singulärwert der Matrix $B$ bezeichnet, d.h. 
      \begin{align*}\sigma_{\min(B)}:=\min\{\lambda^{1/2}\,|\,\lambda\in\sigma(B^*)\}\end{align*}
      mit der adjungierten $B^*$ von $B$.
    \item Das $\varepsilon$-Pseudospektrum $\sigma_\varepsilon(T)$ einer Matrix $T\in\C  ^{n\times n}$ ist 
      invariant unter Orhtonormalen Transformationen, d.h. für eine unitäre Matrix $Q\in\C  ^{n\times n}$ 
      gilt $\sigma_\varepsilon(Q^*TQ) = \sigma_\varepsilon(T)$
  \end{enumerate}
\end{thmbox}
\textit{Beweis.} 
\begin{enumerate}
  \item Es gilt
    \begin{align*}
      \vertn{2}{(zI-T)^{-1}}_2 
      &= \max\{\mu{1/2}\,|\, \mu \text{ Singulärwert von } (zI-T)^{-1}\}  \\
      &= \min\{\mu{1/2}\,|\, \mu \text{ Singulärwert von } (zI-T)\}^{-1} \\
      &= \sigma_{\min}(zI-T)^{-1}
    \end{align*}
    Daraus folgt:
    \begin{align*}
      \sigma_\varepsilon(T) 
      &= \left\{ z \in \C   \;\middle|\; \vertn{2}{(zI-T)^{-1}}_2 \geq \varepsilon^{-1} \right\} \\
      &= \left\{ z \in \C   \;\middle|\; \sigma_{\min}(zI-T)^{-1} \geq \varepsilon^{-1} \right\} \\
      &= \left\{ z \in \C   \;\middle|\; \sigma_{\min}(zI-T) \leq \varepsilon \right\}
    \end{align*}
  \item Übungsaufgabe
\end{enumerate}

Betrachtung einer Folge von Gitterpunkten $z_i\in D\subset \C  $ für $i=1,2,3,\dots$ und in jedem Gitterpunkt
wir das kleinste $\varepsilon$ bestimmt, für das $z_i\in\sigma_\varepsilon(T)$. \\
Interpolation der erhaltenen Werte, bestimmt ob ein Punkte $z\in\C  $ approxmativ zu $\sigma_\varepsilon(T)$ gehört.
